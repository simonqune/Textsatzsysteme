%%%%%%%%Präambel%%%%%%%%%%%%

\documentclass[conference]{IEEEtran}
\IEEEoverridecommandlockouts
%immer rechts kapitel beginnen

%Default-einstellungen für das Dokument

\usepackage[T1]{fontenc}	%Schriftcodierung
\usepackage[utf8]{inputenc}	%Dateicodierung
\usepackage[ngerman]{babel} %Sprachpaket(Silbentrennung,überschriften, Abbildungen)
\usepackage{graphicx}
\usepackage{lmodern} %Vektorschrift
\usepackage{sansmath}
\usepackage{amsfonts}
%\usepackage[margin=1in]{geometry}
\usepackage{amsmath}
\usepackage{blindtext}
\usepackage{svg}
\usepackage{float}
\usepackage{tikz}
\usepackage{pgfplots}
\usepackage{caption}
\captionsetup{
	figurename=Abb:,
}

%Literatur anpassen
\usepackage[
	backend=biber,
	style=numeric-verb,
	sorting=none % numeric-verb, numeric-comb
]{biblatex}
\addbibresource{ref/ref_liste.bib}

%Schriftanpassungen
\renewcommand{\familydefault}{\sfdefault} %serifenlose Schrift
\sansmath			%Aktivieren von Paket "Sansmath"

\usepackage[
	automark, %automatische Kolumne
]{scrlayer-scrpage}
\pagestyle{scrheadings}
\clearscrheadfoot

\ohead[]{\headmark} %dynamische kolumne in kopfzeile
\ofoot[\pagemark]{\pagemark} %Seitenzahl in Fußzeile außen

%wichtige Pakete
\graphicspath{{bilder/svg}{}}  %Pfad zu den Bildern. mehrere pfade mit {{}{}{}} 

\title{ Wie kann der Einsatz von Augmented Reality in der Industrie zu neuen Geschäftsmodellen führen?}

\author{
	\IEEEauthorblockN{Simon Kuhn}
	\IEEEauthorblockA{
		\textit{Technische Hochschule Ingolstadt} \\
		16. Juni 2023 %\today
	}
}

%%%%%%%DOKUMENT%%%%%%%%%%%%%
\begin{document}
\maketitle
\begin{abstract}
	kkkDieser Artikel untersucht die wachsende Rolle von Augmented Reality (AR) im Industriesektor
	und konzentriert sich auf ihr Potenzial zur Optimierung von Betriebsabläufen,
	Steigerung der Produktivität und Förderung innovativer Geschäftsmodelle.
	Dabei wird die technische Grundlage von AR-Systemen eingehend untersucht,
	einschließlich Sensoren und Tracking-Technologien, Datenverarbeitung und
	-darstellung sowie Display-Optionen. Der Artikel diskutiert auch die gängige
	Verwendung von optischen Tracking-Methoden und Inertial Measurement Units (IMUs)
	zur Bestimmung der Position und Ausrichtung von AR-Geräten. Die Bedeutung der
	Datenverarbeitung und -darstellung für die nahtlose Integration von virtuellen
	Inhalten in die reale Welt wird hervorgehoben, zusammen mit einem Überblick über
	verschiedene Display-Optionen wie video-basierte oder optische Displays. Der Artikel
	schließt mit einer Diskussion über die aktuellen Anwendungen von AR in der Industrie,
	einschließlich Mitarbeiterschulungen, Wartung und Produktentwicklung.
\end{abstract}
\begin{IEEEkeywords}
	Augmented Reality, AR, Industrie, Potentiale von AR, Geschäftsmodelle
\end{IEEEkeywords}
\input{kapitel/Einführung}
\section{Technische Grundlagen der Augmented Reality}

Die Grundlage eines funktionsfähigen Augmented Reality-Systems bildet ein
vielfältiges Zusammenspiel verschiedener technischer Systeme und Verfahren.
Diese können grob in folgende Bereiche unterteilt werden, die zusammenarbeiten,
um eine nahtlose und immersive AR-Erfahrung zu ermöglichen.

\subsection{Sensoren und Erfassungstechnologien}

Um virtuelle Inhalte nahtlos in die reale Welt zu integrieren, ist eine präzise
Bestimmung der Position des AR-Systems im Raum erforderlich. Dazu werden
verschiedene Sensoren und Erfassungstechnologien eingesetzt, die die Umgebung
analysieren und die Position sowie Ausrichtung des AR-Geräts ermitteln. Häufig
werden optische Tracking-Verfahren wie Infrarot-Sensoren, 3D-Kameras und
herkömmliche Kameras im sichtbaren Lichtspektrum aufgrund ihrer geringen Kosten
und hohen Verfügbarkeit verwendet. Diese Sensoren erfassen visuelle
Informationen aus der Umgebung und dienen zur präzisen Bestimmung der Position
und Ausrichtung des AR-Geräts anhand von Markern oder speziellen Merkmalen.
Besonders beliebt sind in den letzten Jahren 3D-Kamerasysteme, die auf
structured Light oder Time of Flight basieren. Durch fortschrittliche
Bildverarbeitungsalgorithmen können visuelle Elemente genau erkannt und
verfolgt werden, um virtuelle Objekte in Echtzeit in die reale Welt zu
integrieren. Für den Außenbereich wird häufig auch GPS (Global Positioning
System) verwendet, wobei die Genauigkeit je nach Anwendungsbereich moderat sein
kann und Herausforderungen mit sich bringen kann.

Ein weiterer wichtiger Bestandteil im Bereich der AR sind Inertiale
Messeinheiten (IMUs). Diese Sensoren messen Beschleunigung,
Winkelgeschwindigkeit und magnetische Felder. Durch die Integration von IMUs in
AR-Geräte können Bewegungen und Rotationen des Geräts erfasst und verfolgt
werden, um die relative Position im Raum zu bestimmen. IMUs sind unempfindlich
gegenüber äußeren Störeinflüssen und erfordern keine direkte Sichtlinie zu
anderen Sensoren. Allerdings besteht das Problem, dass sich die gemessene
Position und Ausrichtung im Laufe der Zeit verschieben und von der
tatsächlichen Position abweichen können. Daher werden IMUs häufig in
Kombination mit anderen Sensoren wie 3D-Kameras verwendet, um präzisere
Ergebnisse zu erzielen. Eine erfolgreiche Tracking-Methode basiert grundlegend
auf der Kombination verschiedener Tracking-Verfahren und deren Vorteilen, was
als Sensorfusion bekannt ist. \cite{billinghurst2015survey}
\subsection{Sensorfusion}
Sensorfusion im Bereich der erweiterten Realität bezieht sich auf die
kombinierte Verwendung mehrerer Sensoren, um genaue und zuverlässige
Informationen über die Position und Ausrichtung eines AR-Geräts in der realen
Welt zu erhalten. Ein häufig verwendeter Ansatz für die Sensorfusion ist das
erweiterte Kalman-Filter (EKF), das auf dem Kalman-Filter basiert und speziell
für nichtlineare Systeme entwickelt wurde.

Der EKF verwendet mathematische Modelle, um die Bewegung und das Verhalten der
Sensoren zu beschreiben. Durch die Kombination der Messungen aus verschiedenen
Sensoren werden die Vorteile jedes Sensors genutzt, um eine robuste und genaue
Schätzung der Position und Ausrichtung zu erhalten. Dabei werden die
Unsicherheiten und Fehler der einzelnen Sensoren berücksichtigt und korrigiert.

Die Funktionsweise des EKF beruht auf einer Vorhersage- und einer
Aktualisierungsphase. In der Vorhersagephase werden die aktuellen Schätzungen
der Position und Ausrichtung basierend auf den vorherigen Messungen und dem
mathematischen Modell der Bewegung des AR-Geräts prognostiziert. In der
Aktualisierungsphase werden die aktuellen Messungen der Sensoren verwendet, um
die Vorhersagen zu korrigieren und eine genauere Schätzung zu erhalten.

Ein wichtiger Aspekt des EKF ist die Berücksichtigung der Unsicherheit der
Sensormessungen und des mathematischen Modells. Durch die Anpassung der
Gewichtung der einzelnen Messungen und die Schätzung der Unsicherheit wird eine
robuste Fusion der Sensordaten erreicht.

Die Sensorfusion mithilfe des EKF ermöglicht es, genaue und zuverlässige
Informationen über die Position und Ausrichtung eines AR-Geräts in Echtzeit zu
erhalten. Dies ist entscheidend für eine nahtlose Integration von virtuellen
Inhalten in die reale Welt und ermöglicht eine immersive AR-Erfahrung für die
Benutzer. \cite{5336489,simon2006optimal}


Die mathematischen Gleichungen des EKF sind wie folgt definiert:
\begin{enumerate}
      \item Vorhersage des Zustands:
            \begin{equation*}
                  \hat{x}_{k+1} = g({x}_{k}, u)\
            \end{equation*}
            wobei $g$ die Zustandsübergangsfunktion,
            $x_{k}$ der geschätzte Zustand zum Zeitpunkt $k$ 
            und $u$ die Eingabe zum Zeitpunkt $k$ ist.\\
      \item Vorhersage der Fehlerkovarianz:
            \begin{equation*}
                  P_{k + 1} = J_{A} P_{k} A_{A}^T + Q\
            \end{equation*}
            wobei $J_{A}$ die Jacobi-Matrix der Zustandsübergangsfunktion,
            $P_{k}$ die Kovarianzmatrix des Schätzfehlers 
            und $Q$ die Kovarianzmatrix des Prozessrauschens ist.\\
      \item Berechnung der Kalman Verstärkung:
            \begin{equation*}
                  K_{k} = P_{k} J_{H}^T (J_{H} P_{k} J_{H}^T + R)^{-1}\
            \end{equation*}
            wobei $K_{k} $ die Kalman-Verstärkung,
            $J_{H}$ die Jacobi-Matrix der Messfunktion,
            $R$ die Kovarianzmatrix des Messrauschens und
            $z_{k}$ die Messung zum Zeitpunkt $k$ ist.\\
      \item Aktualisierung der Schätzung durch Messung:
            \begin{equation*}
                  \hat{x}_{k} = \hat{x}_{k} + K_{z} (z_{k} - h (\hat{x}_{k})\
            \end{equation*}
            wobei $J_{H}$ die Jacobi-Matrix der Messfunktion,
            $R$ die Kovarianzmatrix des Messrauschens
            und $z_{k}$ die Messung zum Zeitpunkt $k$ ist.\\

      \item Korrektur der Fehlerkovarianz:
            \begin{equation*}
                  P_{k} = (I - K_{k} J_{H}) P_{k}\
            \end{equation*}
            wobei $I$ die Einheitsmatrix ist.\\
\end{enumerate}

\subsection{Datenverarbeitung und -darstellung}
Die Datenverarbeitung und Darstellung spielen eine entscheidende Rolle im
Bereich der AR. Um eine nahtlose Integration von virtuellen
Inhalten in die reale Welt zu ermöglichen, müssen die erfassten Daten zunächst
verarbeitet und interpretiert werden. Anschließend erfolgt die Darstellung der
AR-Inhalte in einer für den Benutzer verständlichen Form. Dabei ist eine
Kalibrierung der Tracking-Systeme, wie beispielsweise der Kamera, erforderlich,
um genaue Positionierungsinformationen zu erhalten.

Die Darstellung der AR-Inhalte erfolgt in Echtzeit, um eine immersive und
interaktive Erfahrung zu gewährleisten. Hierbei spielen Grafiktechnologien wie
Computergrafik, Rendering-Algorithmen und Schattenberechnung eine wichtige Rolle. Die
virtuellen Objekte müssen realistisch und überzeugend in die reale Umgebung
integriert werden. Dies erfordert die Berücksichtigung von Aspekten wie
Beleuchtung, Schatten und Perspektive, um eine konsistente und immersive
AR-Erfahrung zu schaffen.

Es gibt verschiedene Arten, wie die Darstellung der AR-Inhalte erfolgen kann.
Bei der Video-gestützten Variante wird die Kamera des AR-Geräts verwendet, um
eine Echtzeit-Videoaufnahme der Umgebung zu erfassen und auf dem Display
anzuzeigen. Die AR-Inhalte werden mithilfe von Bildverarbeitungstechniken in
die realen Aufnahmen integriert. Der schematische Aufbau eines solchen
Verfahrens ist in Abbildung \ref{fig:VBAR} dargestellt.\\

\begin{figure}[h]
      \centering
      \includesvg[width=0.7\columnwidth]{bilder/svg/strucutre_video_based_AR.svg}
      \caption[width=0.7\columnwidth]{Aufbau von Video basierten AR-Displays \cite{billinghurst2015survey}}
      \label{fig:VBAR}
\end{figure}

Ein weiteres Verfahren, das in Abbildung 2 dargestellt ist, basiert auf
optischen, durchsichtigen AR-Displays, die auf halbdurchlässigen Spiegeln
basieren. Bei dieser Variante kann der Benutzer die reale Welt durch den
halbdurchlässigen Spiegel sehen und gleichzeitig die reflektierte Anzeige der
AR-Inhalte wahrnehmen. Die Besonderheit dieser Technologie liegt in der
Anordnung des halbdurchlässigen Spiegels unter einem Winkel von 45 Grad.
Dadurch wird das einfallende Licht sowohl reflektiert als auch teilweise
durchgelassen. Ein bekanntes Anwendungsbeispiel für diese Technologie sind
Head-up-Displays in Autos, bei denen relevante Informationen wie
Geschwindigkeit oder Navigationshinweise direkt auf die Windschutzscheibe
projiziert werden. Diese Art der Darstellung ermöglicht es dem Benutzer, die
AR-Inhalte zu sehen, ohne den Blick von der Straße abwenden zu müssen. Auch
erste Versuche von AR-Brillen, wie den Google Glass basierten auf dieser
Technologie.\\

\begin{figure}[h]
      \centering
      \includesvg[width=0.7\columnwidth]{bilder/svg/optical_see_trough_display.svg}
      \caption[width=0.7\columnwidth]{Aufbau von optischen, durchsichtigen AR-Displays \cite{billinghurst2015survey}}
      \label{fig:OSTAR}
\end{figure}

Eine weitere Variante nutzt einen Projektor, um Texturen oder Bilder auf
bestehende Objekte zu projizieren und somit die realen Objekte zu erweitern.
Durch diese Methode werden zusätzliche visuelle Informationen auf die
Oberfläche der Objekte übertragen, um beispielsweise Anleitungen oder virtuelle
Beschriftungen darzustellen.

Diese verschiedenen Technologien der AR-Darstellung können auf unterschiedliche
Arten für den Benutzer zugänglich gemacht werden. Eine häufig verwendete Form
sind sogenannte Head-Mounted-Displays, wie AR-Brillen. Diese ermöglichen es dem
Benutzer, die AR-Inhalte direkt vor seinen Augen wahrzunehmen und sie nahtlos
in die reale Welt einzubetten. Durch das Tragen der AR-Brille hat der Benutzer
die Freiheit, sich in der Umgebung zu bewegen und die AR-Erfahrung ohne
Einschränkungen zu erleben. \cite{billinghurst2015survey}

\subsection{Benutzerschnittstellen und Interaktion}
Die Benutzerschnittstellen und Interaktion spielen eine zentrale Rolle im
Bereich der Augmented Reality und tragen maßgeblich zur intuitiven
Bedienbarkeit, Benutzerfreundlichkeit und Immersion von AR-Anwendungen bei.

Head-Mounted Display (HMD) ermöglichen es dem Benutzer, die virtuellen Inhalte
direkt vor seinen Augen zu sehen. Das HMD kann mit Sensoren ausgestattet sein,
um die Umgebung, sowie Bewegungen des Nutzers zu erkennen. Dies ermöglicht eine
Reihe von Interaktionsmöglichkeiten mit der AR-Anwendung.
\begin{itemize}
      \item 2D User Interfaces: Bei dieser Form der Interaktion,
            werden physische Knöpfe und Tasten verwendet, um mit den virtuellen Inhalten zu interagieren.
            Dies umfasst beispielsweise das Auswählen von Objekten, das Ausführen von Aktionen oder das Navigieren in Menüs. Auch Touch-Eingaben sind möglich.

      \item Für eine erweiterte Interaktionsmöglichkeit werden 3D-Benutzerschnittstellen
            verwendet, die Manipulationen an Objekten mit sechs Freiheitsgraden
            ermöglichen. Dies umfasst das Bewegen, Drehen und Skalieren von Objekten. Die
            Eingabegeräte selbst können unterschiedliche Formen annehmen, wie
            beispielsweise 3D-Mäuse oder Stäbe.

      \item Gestensteuerung: Bei dieser fortschrittlichen Form der Interaktion werden
            Gesten, Handbewegungen und auch Kopfbewegungen verwendet, um mit den virtuellen
            Inhalten zu interagieren. Dies umfasst das Zeigen auf Objekte, das Ziehen und
            Drehen von Objekten sowie das Ausführen von Gesten wie Pinch-to-Zoom. Darüber
            hinaus können Kopfbewegungen zur Steuerung von Menüs, zum Navigieren durch
            virtuelle Umgebungen oder zum Anpassen von Blickwinkeln genutzt werden. Durch
            die Einbindung von Kopfbewegungen in die Gestensteuerung wird eine natürlichere
            und immersivere Interaktion mit den AR-Inhalten ermöglicht. So kann der
            Benutzer beispielsweise durch Drehen des Kopfes seine Perspektive in der
            erweiterten Realität ändern oder durch Kopfbewegungen Menüoptionen auswählen.
            Diese erweiterte Form der Gestensteuerung trägt dazu bei, die Interaktion mit
            AR-Inhalten intuitiver und realitätsnäher zu gestalten. Auch Eye-Tracking ist
            hierbei möglich, um Objekte durch das Ansehen auszuwählen.

      \item Sprachbefehle: Auch Sprachbefehle können genutz werden, um mit Objekten im Raum
            zu interagieren. Benutzer können bestimmte Sprachbefehle verwenden, um Aktionen
            auszuführen, Objekte zu steuern oder Informationen abzurufen. Diese Art der
            Interaktion kann besonders nützlich sein, wenn die Hände des Benutzers
            beschäftigt sind.
\end{itemize}

Die Benutzerschnittstellen und Interaktionsmethoden in der AR werden
kontinuierlich weiterentwickelt, um die Benutzererfahrung zu verbessern und
neue Möglichkeiten der Interaktion zu erschließen. Durch die Integration von
fortgeschrittenen Technologien wie maschinellem Lernen und Künstlicher
Intelligenz ist es möglich, natürlichere und kontextbezogene Interaktionen zu
ermöglichen. Die Gestaltung der Benutzerschnittstellen und Interaktionsmethoden
spielt eine entscheidende Rolle, um AR-Anwendungen zugänglich,
benutzerfreundlich und ansprechend zu gestalten und den Benutzern ein
immersives und interaktives Erlebnis zu bieten. \cite{billinghurst2015survey}\\
\section{AR in der Industrie: Stand der Technik}

AR hat sich in den letzten Jahren zu einer vielversprechenden Technologie in
verschiedenen Branchen entwickelt, insbesondere in der Industrie. Unternehmen
setzen AR erfolgreich ein, um die Effizienz zu steigern, Fehler zu reduzieren
und die Sicherheit am Arbeitsplatz zu verbessern. Dabei findet AR besonders in
den nachfolgenden Bereichen Anwendung.

\subsection{Wartung und Instandhaltung}
Die Integration von AR in den Bereich der Wartung und Instandhaltung hat das
Potenzial, die Effizienz und Genauigkeit dieser Prozesse signifikant zu
verbessern \cite{liu2022probing}. Durch die Nutzung von AR-Brillen oder anderen
AR-Geräten können Techniker während ihrer Arbeit visuelle Informationen und
Anweisungen direkt in ihr Sichtfeld eingeblendet bekommen. Dies ermöglicht eine
schnellere und präzisere Fehlerdiagnose, da relevante Informationen, wie
beispielsweise Schaltpläne, technische Datenblätter oder historische Daten, in
Echtzeit angezeigt werden können. Darüber hinaus können AR-gestützte
Wartungsanleitungen und -simulationen den Technikern helfen, komplexe
Reparaturen oder Wartungsarbeiten durchzuführen, indem sie visuelle
Hilfestellungen und Schritt-für-Schritt-Anleitungen bereitstellen.
\cite{4079262} Dies reduziert das Risiko von Fehlern und verkürzt die
Ausfallzeiten von Maschinen oder Anlagen. Durch die Integration von AR in den
Wartungsprozess können Arbeiter effizienter arbeiten und die cognitive
Arbeitslast reduziert werden \cite{5620905}.

\subsection{Qualitätskontrolle und Inspektion}
Die Anwendung von AR in der Qualitätskontrolle und Inspektion bietet
vielfältige Vorteile, um Prüfprozesse effizienter und präziser zu gestalten. AR
ermöglicht es Inspekteuren, eine optimale Version des fertigen Produktes als
Vergleich einzublenden. Dadurch können Inspekteure Abweichungen oder Mängel
leicht erkennen und bewerten. Dies trägt zur Reduzierung menschlicher Fehler
und zur Verbesserung der Genauigkeit bei der Inspektion bei. Darüber hinaus
kann AR dazu beitragen, den Prozess der Qualitätskontrolle zu beschleunigen.
Hierfür werden potentiell fehlerhafte Bereiche durch Bildverarbeitung oder
Deep-Learning erkannt und dür den Nutzer markiert \cite{9112336}.

\\Forschung
auf dem Gebiet der AR in der Qualitätskontrolle und Inspektion hat gezeigt,
dass die Nutzung dieser Technologie zu einer höheren Inspektionsgenauigkeit,
einer schnelleren Fehlererkennung und einer verbesserten Effizienz führen kann.
Die Integration von AR in diesen Bereich bietet somit großes Potenzial, die
Qualitätssicherung in verschiedenen Industriezweigen zu optimieren und die
Inspektionsprozesse zu verbessern.\cite{etonam2019augmented}
\section{Forschungsbereiche}
Die Forschung im Bereich der Augmented Reality (AR) umfasst eine Vielzahl von
Disziplinen und Forschungsbereichen, die darauf abzielen, die Technologie,
Anwendungen und Interaktionsmöglichkeiten von AR weiterzuentwickeln.

\subsection{Tracking}
Tracking ist ein wichtiger Bereich der AR-Forschung, da es eine der zentralen
Technologien für die Umsetzung von AR-Erfahrungen ist. Es wurden verschiedene
Tracking-Systeme entwickelt, angefangen von einfachen Marker-basierten Systemen
bis hin zu natürlichen Merkmalen und hybriden Sensormethoden. Dennoch sind
weitere Fortschritte erforderlich, um das Ziel einer umfassenden ''Anywhere
Augmentation'' zu erreichen, bei der Nutzer eine überzeugende AR-Erfahrung in
jeder Umgebung haben können.

Für die zuverlässige Outdoor-Augmented-Reality sind Tracking-Methoden
erforderlich, die eine genaue Standortbestimmung über große Flächen
ermöglichen. GPS kann zur Positionsbestimmung verwendet werden, liefert jedoch
in mobilen Geräten der Verbraucher nur eine durchschnittliche Genauigkeit von
5-10 Metern. Eine alternative Methode besteht darin, computergestützte
Bildverarbeitungstechniken in Verbindung mit GPS und inertialen Sensoren zu
verwenden, um die Kameraposition relativ zu bekannten visuellen Merkmalen
abzuschätzen. Diese Methode ist jedoch schwierig auf großflächiges Tracking
anwendbar.

Ein vielversprechender Ansatz für weitreichendes Tracking in unvorbereiteter
Außenumgebung ist die Kombination von Panoramabildern zur Erstellung eines
Punktewolkenmodells. Dieses Modell kann für die positionsbasierte Lokalisierung
über einen Remote-Server und für die Echtzeitverfolgung auf einem mobilen Gerät
verwendet werden. Die Genauigkeit der Verfolgung beträgt weniger als 25 cm
Fehler bei der Positionsbestimmung und unter 0,5 Grad Fehler bei der Rotation.

Für das präzise Indoor-Tracking werden verschiedene Methoden untersucht, wie
Ultraschall, Kameras mit Laser-Trackern, Marker-basiertes Tracking und hybride
Systeme mit Computer Vision, inertialen Sensoren und Ultra-Wideband-Tracking.
Ein vielversprechender Forschungsansatz ist die Verwendung von
Handheld-Tiefensensoren wie dem Google Tango-Projekt. Dies ermöglicht präzise
Innenraumpositionierung und bietet eine ideale Plattform für Indoor-AR.

Eine Herausforderung besteht darin, nahtlos zwischen Outdoor- und
Indoor-Tracking-Umgebungen zu wechseln. Bisher wurden Lösungen entwickelt, die
GPS mit Marker-basiertem Tracking kombinieren oder drahtlose Netzwerke mit GPS
verbinden. Weitere Forschung ist erforderlich, um das Ziel eines
kontinuierlichen und ubiquitären Trackingsystems zu erreichen, das nahtlos
zwischen verschiedenen Umgebungen funktioniert.
\subsection{Interaktion}

In Abschnitt 7 wird ein Überblick über verschiedene Interaktionsmethoden für
Augmented Reality gegeben. Frühe AR-Schnittstellen verwendeten Techniken, die
von Desktop-Schnittstellen oder Virtual Reality inspiriert waren. Im Laufe der
Zeit wurden jedoch innovativere Methoden wie Tangible AR oder natürliche
Gesteninteraktion eingesetzt. Es besteht immer noch erheblicher
Forschungsbedarf in neuen Interaktionsmethoden, insbesondere in den Bereichen
intelligente Systeme, hybride Benutzerschnittstellen und kollaborative Systeme.
In diesem Abschnitt werden einige Möglichkeiten in jedem dieser Bereiche kurz
vorgestellt.

Bisherige Forschungsergebnisse haben gezeigt, dass AR eine sehr natürliche Art
der Interaktion mit virtuellen Inhalten ist, aber in vielen Fällen waren die
Schnittstellen selbst nicht sehr intelligent und reagierten nicht
unterschiedlich auf Benutzereingaben. Das Feld der Intelligent User Interfaces
(IUI) hat sich in den letzten zwanzig Jahren entwickelt, in dem untersucht
wird, wie künstliche Intelligenz mit Methoden der Mensch-Computer-Interaktion
kombiniert werden kann, um reaktionsschnellere Schnittstellen zu erzeugen. Es
gibt jedoch nur wenig Forschung zu IUI-Methoden in AR. Einige Forscher haben
begonnen, den Einsatz von virtuellen Charakteren zu erkunden, die begrenzte
Intelligenz zeigen. Zum Beispiel wurde das Welbo-Interface entwickelt, bei dem
ein Charakter in der realen Welt zu sehen war und auf einfache Sprachbefehle
reagierte. Ein anderes Beispiel ist Mr Virtuoso, eine AR-Schnittstelle, die
einen virtuellen Charakter zur Vermittlung von Kunstwissen einsetzte.

Ein vielversprechender Forschungsbereich ist die intelligente Trainingssysteme
(ITS). Frühere Forschungen haben gezeigt, dass sowohl AR als auch
ITS-Anwendungen das Training erheblich verbessern können. Beispielsweise
ermöglicht AR-Technologie das Überlagern virtueller Hinweise auf die Ausrüstung
von Arbeitern und hilft bei räumlichen Aufgaben. ITS-Anwendungen ermöglichen es
den Menschen, eine auf ihren individuellen Lernstil zugeschnittene
Lernerfahrung zu haben, indem sie intelligente, reaktionsschnelle Rückmeldungen
bieten. Es wurde gezeigt, dass ITS die Lernerfolge um mindestens eine Note
verbessern, das Lernen erheblich beschleunigen und beeindruckende Ergebnisse
bei der Wissensübertragung erzielen können. Es gibt jedoch nur wenig Forschung,
die untersucht, wie beide Technologien kombiniert werden können.

Ein weiterer vielversprechender Forschungsbereich ist die Entwicklung hybrider
AR-Schnittstellen und -Interaktionen. Anfangs waren AR-Systeme eigenständige
Anwendungen, bei denen der Benutzer sich ausschließlich auf die
AR-Schnittstelle und die Interaktion mit den virtuellen Inhalten konzentrierte.
Mit der Zeit wurden jedoch hybride Schnittstellen entwickelt, die AR mit
anderen Interaktionsmethoden kombinieren. Es gibt interessante Möglichkeiten
für Forschung in den Bereichen AR und Ubiquitous Computing, AR und VR sowie AR
und herkömmliche Desktop-Schnittstellen.

Eine weitere Möglichkeit besteht darin, AR mit intelligenten
Benutzerschnittstellen, VR-Schnittstellen und Ubiquitous
Computing-Schnittstellen zu kombinieren. Dies könnte dazu führen, dass VR und
AR nahtlos ineinander übergehen und mit Ubiquitous Computing-Technologien
verschmelzen. Diese hybriden Schnittstellen könnten den Benutzern ständigen
Zugriff auf Informationen ermöglichen und verschiedene
Schnittstellendarstellungen verwenden.

Insgesamt sind die aktuellen AR-Trainingssysteme nicht intelligent, und
aktuelle ITS verwenden keine AR-Schnittstelle für ihre Benutzerschnittstelle.
Es besteht die Notwendigkeit, intelligente AR-Trainingssysteme zu entwickeln,
die Benutzern Feedback zur Qualität ihrer Aufgaben geben. Die Kombination von
AR mit IUI-Methoden und anderen Schnittstellentechnologien eröffnet
interessante Möglichkeiten für weiterführende Forschung.
\subsection{Displays}
Im Bereich der AR-Displaytechnologie hat es seit Ivan Sutherlands erstem System
erhebliche Fortschritte gegeben, aber aktuelle Displays sind immer noch weit
von Sutherlands Vision des "Ultimate Display" entfernt. Es gibt wichtige
Forschungsmöglichkeiten in der Gestaltung von Head-Mounted Displays,
Projektionstechnologie, Kontaktlinsen-Displays und anderen Bereichen.

Traditionelle optische Durchsicht-Displays haben jedoch einige Nachteile, wie
zum Beispiel eine beschränkte Sichtfeldunterstützung und keine echte Verdeckung
der realen Welt. Die ideale Anzeige wäre eine, die ein großes Sichtfeld
unterstützt, Verdeckung der realen Welt ermöglicht und Bilder auf verschiedenen
Fokusebenen liefert, und das alles in einem kleinen und unauffälligen
Formfaktor. Forscher haben in jedem dieser Bereiche Fortschritte gemacht, aber
es gibt noch viel Arbeit zu tun.

Es wurden verschiedene Designs für optische Durchsicht-Displays entwickelt, um
diese Mängel zu beheben. Zum Beispiel haben Kiyokawa et al. untersucht, wie
elektronische Maskierungselemente zu optischen Durchsicht-Displays hinzugefügt
werden können, um die Verdeckungsproblematik zu lösen. Andere Forscher haben
sich mit der Erweiterung des Sichtfeldes und der Variation der Fokusebene
befasst. Es gab auch Versuche, all diese Probleme in einem Design zu lösen.

Eine weitere interessante Forschungsmöglichkeit im Bereich Displays liegt in
der Entwicklung von Head-Mounted Projection Displays (HMPD). Frühe
Projektor-Systeme waren sperrig und nicht tragbar, aber Fortschritte in der
Pico-Projektortechnologie haben diese Einschränkungen überwunden. Es gibt auch
Forschung darüber, wie die Interaktion mit projizierten AR-Inhalten verbessert
werden kann, zum Beispiel durch Gestensteuerung oder den Einsatz physischer
Objekte als Benutzerschnittstelle.

Ein vielversprechender Ansatz sind kontaktlinsenbasierte Displays. Das Ziel ist
es, ein Head-Mounted Display zu entwickeln, das für andere Personen um den
Benutzer herum nicht wahrnehmbar ist. Durch den Einsatz von MEMS-Technologie
und drahtloser Energie- und Datenübertragung könnten aktive Pixel in eine
Kontaktlinse integriert werden. Es wurden bereits Prototypen entwickelt, aber
es gibt noch Herausforderungen wie die Integration von Optiken, ausreichende
Sauerstoffversorgung der Hornhaut und kontinuierliche Stromversorgung und
Datenübertragung, die gelöst werden müssen.
\subsection{Soziale Akzeptanz}

Soziale Akzeptanz ist ein wichtiger Faktor, der die Verbreitung von Augmented
Reality (AR) beeinflusst, insbesondere bei tragbaren oder mobilen Systemen.
Obwohl AR-Systeme in Bezug auf Größe und Gewicht immer kleiner geworden sind,
gibt es immer noch erheblichen Widerstand in der Gesellschaft gegenüber Geräten
wie Google Glass. Eine Umfrage in den USA ergab beispielsweise, dass nur 12
Prozent der Befragten bereit wären, "Augmented-Reality-Brillen" von einer
Marke, der sie vertrauen, zu tragen. Die Gründe für diese Zurückhaltung können
Datenschutzbedenken, die Angst, lächerlich auszusehen, oder die Sorge, zum Ziel
von Dieben zu werden, sein.

Diese Bedenken beschränken sich nicht nur auf tragbare AR-Systeme. Wenn eine
Person beispielsweise mit einem mobilen Telefon oder Tablet durch eine Stadt
geht und eine AR-Browser-Anwendung verwendet, um sich zurechtzufinden oder
AR-Inhalte anzuzeigen, muss sie das Telefon auf Augenhöhe vor sich halten,
während sie geht. Diese unnatürliche Haltung kann sich albern anfühlen und
andere Menschen denken lassen, dass sie gefilmt werden.

Obwohl einige Forscher die soziale Akzeptanz als wichtigen Aspekt von AR
hervorgehoben haben, gab es zunächst wenig Forschung zu diesem Thema. In
jüngerer Zeit wurden jedoch einige Arbeiten durchgeführt. Zum Beispiel wurden
positive Ergebnisse erzielt, als AR-Technologie als Instrument für klinische
Schulungen in einem Krankenhaus eingesetzt wurde. Auch in einer universitären
Umgebung zeigten Studien, dass die Mehrheit der Studenten AR für das Lehren und
Lernen als nützlich erachtete.

Es gibt jedoch nur wenige Untersuchungen zur sozialen Akzeptanz von AR in
öffentlichen oder sozialen Situationen. Es besteht ein Bedarf an weiterer
Forschung auf diesem Gebiet, insbesondere im Hinblick auf die Erfahrung von
Benutzern mit mobilen AR-Diensten und den damit verbundenen sozialen und
emotionalen Aspekten. Bisherige Studien zeigen, dass soziale Akzeptanzprobleme
bei Personen, die ständig eine AR-Brille tragen, wahrscheinlich höher sind als
bei kurzzeitiger Nutzung von mobilen oder tragbaren AR-Systemen.

Es ist zu erwarten, dass sich die soziale Akzeptanz verbessert, wenn
AR-Technologie unauffälliger wird und Displays sowie Eingabegeräte in Kleidung
integriert werden. Die Erforschung der sozialen Akzeptanz in tragbaren oder
mobilen AR-Erlebnissen, insbesondere in öffentlichen Umgebungen, ist ein
wichtiger Bereich für zukünftige Arbeit in der AR-Gemeinschaft.
\section{Potenzial von AR für neue Geschäftsmodelle}

Neben den bestehenden Anwendungsfällen von AR in der Industrie ergeben sich
durch den stetigen technologischen Fortschritt neue potentielle Anwendungsfälle
und Geschäftsmodelle. Der Tabelle 1 sind die häufigsten Suchanfragen im
Zusammenhang mit Augmented Reality zu entnehmen. Abbildung
\ref{fig:studienverteilung} veranschaulicht dabei die Anzahl an Primärstudien
in den verschiedenen Bereichen von Ar. Dies kann einen guten Eindruck über
aktuelle Trends in diesem Bereich geben. Im folgenden Kapitel werden einige der
interessantesten Potenziale von AR in der Industrie dargestellt.\\

\begin{table}[h]
    \centering
    \captionsetup{font=small}
    \label{tabelle1}
    \renewcommand{\arraystretch}{1.35} % Anpassung des Zeilenabstands um den Faktor 1.2

    \begin{tabular}{c|l}
        \multicolumn{1}{c|}{Platzierung} & \multicolumn{1}{c}{\centering Suchanfrage}  \\
        \hline
        1                                & ''Augmented Reality'' AND ''Manufacturing'' \\
        2                                & ''Augmented Reality'' AND ''Production"''   \\
        3                                & ''Augmented Reality'' AND ''Assembly''      \\
        4                                & ''Augmented Reality'' AND ''Shop floor''    \\
        5                                & ''Augmented Reality'' AND ''Factory floor'' \\
        \hline

    \end{tabular}
    \caption{Suchbegriffe im Bezug auf AR in der Industrie \cite{de2018augmented}}

\end{table}

\begin{figure}[h]
    \centering
    \includesvg[width=0.9\columnwidth]{bilder/svg/studienverteilung_ar.svg}
    \caption[width=0.7schaff\columnwidth]{Anzahl der Primärstudien pro potenziellem Einsatzbereich von Augmented Reality in der Industrie \cite{de2020survey}}
    \label{fig:studienverteilung}
\end{figure}

\subsection{Produktdesign und -entwicklung}
AR bietet ein großes Potenzial für den Anwendungsbereich
des Produktdesigns. Durch die Integration von AR-Technologien in den
Designprozess können Designer und Ingenieure hochwertige Vorschauen und
interaktive Simulationen ihrer Produkte erstellen. Die Nutzung einer
cloud-basierten Anwendung ermöglicht Echtzeit-Kollaboration und Meetings
zwischen Designern weltweit, unabhängig von ihrem Standort. Dies fördert die
Zusammenarbeit, verbessert die Kommunikation und ermöglicht präzisere
Entscheidungen während des Designprozesses. Mit AR können virtuelle Prototypen
in der realen Welt platziert werden, um das Aussehen, die Funktionalität und
die Benutzererfahrung zu bewerten. Dies hilft dabei, Designkonflikte und
-probleme frühzeitig zu erkennen und zu lösen. Darüber hinaus ermöglicht AR
eine immersive Darstellung von Produkten, sodass Kunden und Stakeholder sie vor
dem eigentlichen Bau oder der Produktion in einem realistischen Kontext erleben
können. Dies verbessert nicht nur das Verständnis des Produkts, sondern
ermöglicht auch wertvolles Feedback und iteratives Design. Durch die Nutzung
von AR im Produktdesign können Unternehmen die Effizienz steigern, die
Fehlerquote verringern und letztendlich bessere Produkte entwickeln, die den
Bedürfnissen der Kunden entsprechen.\cite{mourtzis2020augmented}

\subsection{Fernwartung und Montage}
Die Nutzung von Augmented Reality zur Unterstützung von Wartungsarbeiten
an industriellen Geräten hat sich als vielversprechender Anwendungsbereich
erwiesen. Durch die Integration von AR in die Fernwartung könnten sich sogar
noch effektivere Möglichkeiten eröffnen. Bereits erste funktionierende Ansätze
zeigen, dass AR eine bedeutende Rolle bei der Durchführung von
Fernwartungsarbeiten spielen kann. \cite{masoni2017supporting} \\Ein weiterer
Schritt in dieser Entwicklung wäre der verstärkte Einsatz von AR in der
Produktmontage. Im Vergleich zu gedruckten Handbüchern bietet AR klare
Vorteile, da sie die Aufmerksamkeit auf das zu bearbeitende Objekt lenkt und
Fehler erkennt und reduziert. Frühere Arbeiten auf diesem Gebiet waren
experimentell und konzentrierten sich auf spezifische Objekte. Dabei wurden
häufig Tracking-Techniken auf Basis von Referenzmarkierungen verwendet. Durch
Fortschritte in Bereichen wie Deep Learning kann AR jedoch zunehmend komplexe
Montageprozesse unterstützen. Dennoch ist es teilweise schwierig, die konkreten
Vorteile von AR in diesem Bereich nachzuweisen, Studien in diesem Bereich
kommen zu unterschiedlichen Ergebnissen \cite{tang2003comparative}. Dies kann
auf verschiedene Faktoren zurückzuführen sein, wie z.B. die Komplexität der
Montageprozesse, die Qualität der eingesetzten AR-Anwendungen und die
Einarbeitung der Mitarbeiter in die Nutzung der Technologie. Trotzdem wird die
kontinuierliche technologische Weiterentwicklung AR in Zukunft noch
leistungsfähiger machen und den Unternehmen noch mehr Möglichkeiten bieten,
komplexe Montageprozesse zu unterstützen und zu optimieren. Es ist zu erwarten,
dass AR in der Industrie einen immer größeren Stellenwert einnehmen wird und zu
einer effektiven Unterstützung bei der Montage von Produkten
wird.\cite{8951930}

\subsection{Marketing}
Die Potenziale von Augmented Reality im Bereich des Marketings sind
vielfältig und vielversprechend. AR ermöglicht es Unternehmen, interaktive und
immersivere Markenerlebnisse für ihre Zielgruppe zu schaffen. Durch die
Einbindung digitaler Inhalte in die reale Welt können Produkte und
Dienstleistungen auf innovative und ansprechende Weise präsentiert werden. AR
bietet die Möglichkeit, Kunden in ihren individuellen Entscheidungskontexten
anzusprechen und personalisierte, relevante Informationen bereitzustellen.
Darüber hinaus eröffnet AR neue Wege der Kundeninteraktion, indem es
spielerische Elemente integriert. Durch die Schaffung von emotionalen Bindungen
und positiven Markenerlebnissen kann AR das Kundenengagement und die
Markenloyalität erhöhen. Unternehmen haben auch die Möglichkeit, AR als
Instrument für datengesteuertes Marketing einzusetzen, indem sie Nutzungsdaten
und Interaktionsmuster erfassen und analysieren, um gezielte Marketingmaßnahmen
abzuleiten. Die kontinuierlichen Fortschritte in der AR-Technologie und die
steigende Verbreitung von AR-fähigen Endgeräten eröffnen spannende Perspektiven
für innovative Marketingstrategien und -kampagnen, die das Kundenerlebnis
revolutionieren können. \cite{chylinski2020augmented,rauschnabel2019augmented}

\subsection{Schulung und Training}
Augmented Reality wird immer wieder im Zusammenhang mit Schulungsprozessen
in der Industrie erwähnt. Besonders interessant ist dabei der Einsatz von
AR-basierten Assistenzsystemen während des Schulungsprozesses. Theoretisch
könnten dadurch der Umgang mit potenziell gefährlichen Situationen in einer
sicheren Umgebung trainiert werden. Auch das Erlernen von Montageaufgaben ist
mit AR-Unterstützung möglich. Dabei können Anleitungen in Papierform durch AR
ersetzt und Fehler beim Aufbau durch den Monteur erkannt und mitgeteilt werden.
Allerdings gibt es in diesem Bereich keine klaren Studien, die den besseren
Lernerfolg durch AR belegen. Am besten schneidet weiterhin die persönliche
Schulung ab. Das AR-Assistenzsystem verhindert effektiv ein fehlerhaftes
Erlernen von Inhalten, erreicht jedoch nicht die Geschwindigkeit der
persönlichen Schulung. Im Vergleich zu einer individuellen Schulung mit einem
gedruckten Handbuch konnte keine höhere Effizienz der AR-Schulung nachgewiesen
werden. Bezüglich der langfristigen Wissensnachhaltigkeit zeigt sich, dass es
keine Unterschiede im Erinnerungsvermögen der Teilnehmer an den Montageprozess
gab, unabhängig von der Schulungsmethode. Die Ergebnisse legen nahe, dass
AR-Assistenzsysteme als hilfreiche Werkzeuge für die Schulung von Arbeitern in
Montageaufgaben eingesetzt werden können, insbesondere um fehlerhaftes Lernen
zu vermeiden. Es ist jedoch zu beachten, dass weitere Vorteile wie eine höhere
Schulungseffizienz in dieser Studie nicht belegt werden konnten. Dennoch ist zu
erwarten, dass die Effizienz der Schulungen durch den technischen Fortschritt
im Bereich AR weiter zunehmen wird und AR-gestützte Schulungen Unternehmen
einen großen Vorteil bieten können.
\section{Fazit}

\printbibliography
\end{document}