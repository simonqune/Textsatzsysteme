%%%%%%%%Präambel%%%%%%%%%%%%

\documentclass[conference]{IEEEtran}
\IEEEoverridecommandlockouts
\DeclareUnicodeCharacter{0301}{\'{e}}

%immer rechts kapitel beginnen

%Default-einstellungen für das Dokument

\usepackage[T1]{fontenc}	%Schriftcodierung
\usepackage[utf8]{inputenc}	%Dateicodierung
\usepackage[ngerman]{babel} %Sprachpaket(Silbentrennung,überschriften, Abbildungen)
\usepackage{graphicx}
\usepackage{lmodern} %Vektorschrift
\usepackage{sansmath}
\usepackage{amsfonts}
%\usepackage[margin=1in]{geometry}
\usepackage{amsmath}
\usepackage{blindtext}
\usepackage{svg}
\usepackage{float}
\usepackage{tikz}
\usepackage{pgfplots}
\usepackage{caption}
\captionsetup{
	figurename=Abb:,
}

%Literatur anpassen
\usepackage[
	backend=biber,
	style=numeric-verb,
	sorting=none % numeric-verb, numeric-comb
]{biblatex}
\addbibresource{ref/ref_liste.bib}

%Schriftanpassungen
\renewcommand{\familydefault}{\sfdefault} %serifenlose Schrift
\sansmath			%Aktivieren von Paket "Sansmath"

\usepackage[
	automark, %automatische Kolumne
]{scrlayer-scrpage}
\pagestyle{scrheadings}
\clearscrheadfoot

\ohead[]{\headmark} %dynamische kolumne in kopfzeile
\ofoot[\pagemark]{\pagemark} %Seitenzahl in Fußzeile außen

%wichtige Pakete
\graphicspath{{bilder/svg}{}}  %Pfad zu den Bildern. mehrere pfade mit {{}{}{}} 

\title{ Wie kann der Einsatz von Augmented Reality in der Industrie zu neuen Geschäftsmodellen führen?}

\author{
	\IEEEauthorblockN{Simon Kuhn}
	\IEEEauthorblockA{
		\textit{Technische Hochschule Ingolstadt} \\
		16. Juni 2023 %\today
	}
}

%%%%%%%DOKUMENT%%%%%%%%%%%%%
\begin{document}
\maketitle
\begin{abstract}
	Dieser Artikel untersucht die wachsende Rolle von Augmented Reality (AR) im Industriesektor
	und konzentriert sich auf ihr Potenzial zur Optimierung von Betriebsabläufen,
	Steigerung der Produktivität und Förderung innovativer Geschäftsmodelle.
	Dabei wird die technische Grundlage von AR-Systemen eingehend untersucht,
	einschließlich Sensoren und Tracking-Technologien, Datenverarbeitung und
	-darstellung sowie Display-Optionen. Der Artikel diskutiert auch die gängige
	Verwendung von optischen Tracking-Methoden und Inertial Measurement Units (IMUs)
	zur Bestimmung der Position und Ausrichtung von AR-Geräten. Die Bedeutung der
	Datenverarbeitung und -darstellung für die nahtlose Integration von virtuellen
	Inhalten in die reale Welt wird hervorgehoben, zusammen mit einem Überblick über
	verschiedene Display-Optionen wie video-basierte oder optische Displays. Der Artikel
	schließt mit einer Diskussion über die aktuellen Anwendungen von AR in der Industrie,
	einschließlich Mitarbeiterschulungen, Wartung und Produktentwicklung.
\end{abstract}
\begin{IEEEkeywords}
	Augmented Reality, AR, Industrie, Potentiale von AR, Geschäftsmodelle
\end{IEEEkeywords}
\section{Einführung}

Dank des technologischen Fortschritts im Bereich der Augmented Reality wurden
enorme Fortschritte erzielt. AR ermöglicht es, digitale Inhalte in die
physische Welt zu projizieren und somit die Realität mit virtuellen
Informationen und Objekten zu erweitern. Diese Technologie eröffnet
weitreichende Möglichkeiten zur Optimierung von Prozessen, Steigerung der
Produktivität und Entwicklung innovativer Geschäftsmodelle in der Industrie.

Der Einsatz von AR in der Industrie ist bereits weit verbreitet und vielfältig.
Er findet Anwendung bei der Wartung und Instandhaltung von Maschinen und
Anlagen sowie bei der Qualitätskontrole. Dabei kommen verschiedene
AR-Technologien und Plattformen zum Einsatz, wie beispielsweise Head-mounted
Displays (HMDs), Smart Glasses oder markerbasierte AR-Systeme. Diese
Technologien bieten den Benutzern ein immersives AR-Erlebnis und ermöglichen
eine direkte Interaktion mit den virtuellen Inhalten in der realen Umgebung.
\section{Technische Grundlagen der Augmented Reality}
Die Grundlage eines funktionsfähigen Augmented Reality-Systems bilden
verschiedene technische Systeme und Verfahren. Diese lassen sich in folgende
drei Bereiche unterteilen.

\subsection{Sensoren und Erfassungstechnologien}

Um virtuelle Inhalte nahtlos in die reale Welt zu integrieren, ist eine präzise
Bestimmung der Position des AR-Systems im Raum erforderlich. Dazu werden
verschiedene Sensoren und Erfassungstechnologien eingesetzt, die die Umgebung
analysieren und die Position sowie Ausrichtung des AR-Geräts ermitteln.
Optische Tracking-Verfahren wie Infrarot-Sensoren, 3D-Kameras und herkömmliche
Kameras im sichtbaren Lichtspektrum werden aufgrund ihrer geringen Kosten und
hohen Verfügbarkeit häufig verwendet. Sie erfassen visuelle Informationen aus
der Umgebung und dienen zur Erkennung von Markern oder speziellen Merkmalen, um
die Position und Ausrichtung des AR-Geräts präzise zu bestimmen. Insbesondere
3D-Kamerasysteme, die auf structured Light oder Time of Flight basieren, sind
in den letzten Jahren immer beliebter geworden. Mithilfe fortschrittlicher
Bildverarbeitungsalgorithmen erfolgt eine genaue Erkennung und Verfolgung
visueller Elemente, um virtuelle Objekte in Echtzeit in die reale Welt zu
integrieren.

Ein weiterer wichtiger Bestandteil im Bereich der AR sind Inertiale
Messeinheiten (IMUs). Diese Sensoren messen Beschleunigung,
Winkelgeschwindigkeit und magnetische Felder. Durch die Integration von IMUs in
AR-Geräte können Bewegungen und Rotationen des Geräts erfasst und verfolgt
werden, um die relative Position im Raum zu bestimmen. IMUs sind unempfindlich
gegenüber äußeren Störeinflüssen und erfordern keine direkte Sichtlinie zu
anderen Sensoren. Es besteht jedoch das Problem, dass sich die gemessene
Position und Ausrichtung im Laufe der Zeit verschiebt und von der tatsächlichen
Position abweicht. Daher werden IMUs häufig in Kombination mit anderen Sensoren
wie 3D-Kameras verwendet, um präzisere Ergebnisse zu erzielen.

\subsection{Datenverarbeitung und -darstellung}

Die Datenverarbeitung und Darstellung spielen eine entscheidende Rolle im
Bereich der Augmented Reality (AR). Um eine nahtlose Integration von virtuellen
Inhalten in die reale Welt zu ermöglichen, müssen die erfassten Daten zunächst
verarbeitet und interpretiert werden. Anschließend erfolgt die Darstellung der
AR-Inhalte in einer für den Benutzer verständlichen Form. Dabei ist eine
Kalibrierung der Tracking-Systeme, wie beispielsweise der Kamera, erforderlich,
um genaue Positionierungsinformationen zu erhalten.

Die Darstellung der AR-Inhalte erfolgt in Echtzeit, um eine immersive und
interaktive Erfahrung zu gewährleisten. Hierbei spielen Grafiktechnologien wie
Computergrafik, Rendering-Algorithmen und Shading eine wichtige Rolle. Die
virtuellen Objekte müssen realistisch und überzeugend in die reale Umgebung
integriert werden. Dies erfordert die Berücksichtigung von Aspekten wie
Beleuchtung, Schatten und Perspektive, um eine konsistente und immersive
AR-Erfahrung zu schaffen.

Die Darstellung der AR-Inhalte kann auf verschiedene Arten erfolgen. Bei der
video-gestützten Variante wird die Kamera des AR-Geräts verwendet, um eine
Echtzeit-Videoaufnahme der Umgebung zu erfassen und auf dem Display
darzustellen. Die AR-Inhalte werden dabei durch Bildverarbeitungstechniken in
die realen Aufnahmen integriert. Bei der optischen Variante wird ein
transparentes Display verwendet, um die virtuellen Inhalte direkt in das
Sichtfeld des Benutzers zu projizieren.

Eine andere Variante sind optische, durchsichtige AR-Displays, die auf
halbdurchlässigen Spiegeln basieren. Der Benutzer kann dabei die reale Welt
durch den halbdurchlässigen Spiegel sehen und gleichzeitig die reflektierte
Anzeige der AR-Inhalte wahrnehmen. Ein bekanntes Anwendungsbeispiel für diese
Technologie sind Head-up-Displays in Autos.

Eine weitere Variante nutzt einen Projektor, um Texturen oder Bilder auf
bestehende Objekte zu projizieren und so die realen Objekte zu erweitern.

Diese verschiedenen Technologien der AR-Darstellung können in verschiedenen
Formen für den Benutzer zugänglich gemacht werden. Häufig geschieht dies in
Form von sogenannten Head-Mounted-Displays, wie beispielsweise AR-Brillen.
Diese ermöglichen es dem Benutzer, die AR-Inhalte direkt vor seinen Augen
wahrzunehmen und in die reale Welt einzubetten.

\subsection{Benutzerschnittstellen und Interaktion}
Die Benutzerschnittstellen und Interaktion spielen eine zentrale Rolle im
Bereich der Augmented Reality (AR) und tragen maßgeblich zur intuitiven
Bedienbarkeit, Benutzerfreundlichkeit und Immersion von AR-Anwendungen bei.

Eine der gängigsten Formen der Benutzerschnittstelle in AR-Anwendungen ist das
Head-Mounted Display (HMD), das dem Benutzer ermöglicht, die virtuellen Inhalte
direkt vor seinen Augen zu sehen. Das HMD kann mit Sensoren ausgestattet sein,
um die Umgebung, sowie Bewegungen des Nutzers zu erkennen. Dies ermöglicht eine
Reihe von Interaktionsmöglichkeiten mit der AR-Anwendung.
\begin{itemize}
      \item 2D User Interfaces: Bei dieser Form der Interaktion,
            werden physische Knöpfe und Tasten verwendet, um mit den virtuellen Inhalten zu interagieren.
            Dies umfasst beispielsweise das Auswählen von Objekten, das Ausführen von Aktionen oder das Navigieren in Menüs. Auch Touch-Eingaben sind möglich.

      \item Für eine erweiterte Interaktionsmöglichkeit werden 3D-Benutzerschnittstellen
            verwendet, die Manipulationen an Objekten mit sechs Freiheitsgraden
            ermöglichen. Dies umfasst das Bewegen, Drehen und Skalieren von Objekten. Die
            Eingabegeräte selbst können unterschiedliche Formen annehmen, wie
            beispielsweise 3D-Mäuse oder Stäbe.

      \item Gestensteuerung: Bei dieser fortschrittlichen Form der Interaktion werden
            Gesten, Handbewegungen und auch Kopfbewegungen verwendet, um mit den virtuellen
            Inhalten zu interagieren. Dies umfasst das Zeigen auf Objekte, das Ziehen und
            Drehen von Objekten sowie das Ausführen von Gesten wie Pinch-to-Zoom. Darüber
            hinaus können Kopfbewegungen zur Steuerung von Menüs, zum Navigieren durch
            virtuelle Umgebungen oder zum Anpassen von Blickwinkeln genutzt werden. Durch
            die Einbindung von Kopfbewegungen in die Gestensteuerung wird eine natürlichere
            und immersivere Interaktion mit den AR-Inhalten ermöglicht. So kann der
            Benutzer beispielsweise durch Drehen des Kopfes seine Perspektive in der
            erweiterten Realität ändern oder durch Kopfbewegungen Menüoptionen auswählen.
            Diese erweiterte Form der Gestensteuerung trägt dazu bei, die Interaktion mit
            AR-Inhalten intuitiver und realitätsnäher zu gestalten. Auch Eye-Tracking ist
            hierbei möglich, um Objekte durch das Ansehen auszuwählen.

      \item Sprachbefehle: Auch Sprachbefehle können genutz werden, um mit Objekten im Raum
            zu interagieren. Benutzer können bestimmte Sprachbefehle verwenden, um Aktionen
            auszuführen, Objekte zu steuern oder Informationen abzurufen. Diese Art der
            Interaktion kann besonders nützlich sein, wenn die Hände des Benutzers
            beschäftigt sind.
\end{itemize}

Die Benutzerschnittstellen und Interaktionsmethoden in der AR werden
kontinuierlich weiterentwickelt, um die Benutzererfahrung zu verbessern und
neue Möglichkeiten der Interaktion zu erschließen. Durch die Integration von
fortgeschrittenen Technologien wie maschinellem Lernen und Künstlicher
Intelligenz ist es möglich, natürlichere und kontextbezogene Interaktionen zu
ermöglichen. Die Gestaltung der Benutzerschnittstellen und Interaktionsmethoden
spielt eine entscheidende Rolle, um AR-Anwendungen zugänglich,
benutzerfreundlich und ansprechend zu gestalten und den Benutzern ein
immersives und interaktives Erlebnis zu bieten.
\section{AR in der Industrie: Stand der Technik}

AR hat sich in den letzten Jahren zu einer vielversprechenden Technologie in
verschiedenen Branchen entwickelt, insbesondere in der Industrie. Unternehmen
setzen AR erfolgreich ein, um die Effizienz zu steigern, Fehler zu reduzieren
und die Sicherheit am Arbeitsplatz zu verbessern. Dabei findet AR besonders in
den nachfolgenden Bereichen Anwendung.

\subsection{Wartung und Instandhaltung}
Die Integration von AR in den Bereich der Wartung und Instandhaltung hat das
Potenzial, die Effizienz und Genauigkeit dieser Prozesse signifikant zu
verbessern \cite{liu2022probing}. Durch die Nutzung von AR-Brillen oder anderen
AR-Geräten können Techniker während ihrer Arbeit visuelle Informationen und
Anweisungen direkt in ihr Sichtfeld eingeblendet bekommen. Dies ermöglicht eine
schnellere und präzisere Fehlerdiagnose, da relevante Informationen, wie
beispielsweise Schaltpläne, technische Datenblätter oder historische Daten, in
Echtzeit angezeigt werden können. Darüber hinaus können AR-gestützte
Wartungsanleitungen und -simulationen den Technikern helfen, komplexe
Reparaturen oder Wartungsarbeiten durchzuführen, indem sie visuelle
Hilfestellungen und Schritt-für-Schritt-Anleitungen bereitstellen.
\cite{4079262} Dies reduziert das Risiko von Fehlern und verkürzt die
Ausfallzeiten von Maschinen oder Anlagen. Durch die Integration von AR in den
Wartungsprozess können Arbeiter effizienter arbeiten und die cognitive
Arbeitslast reduziert werden \cite{5620905}.

\subsection{Qualitätskontrolle und Inspektion}
Die Anwendung von AR in der Qualitätskontrolle und Inspektion bietet
vielfältige Vorteile, um Prüfprozesse effizienter und präziser zu gestalten. AR
ermöglicht es Inspekteuren, eine optimale Version des fertigen Produktes als
Vergleich einzublenden. Dadurch können Inspekteure Abweichungen oder Mängel
leicht erkennen und bewerten. Dies trägt zur Reduzierung menschlicher Fehler
und zur Verbesserung der Genauigkeit bei der Inspektion bei. Darüber hinaus
kann AR dazu beitragen, den Prozess der Qualitätskontrolle zu beschleunigen.
Hierfür werden potentiell fehlerhafte Bereiche durch Bildverarbeitung oder
Deep-Learning erkannt und dür den Nutzer markiert \cite{9112336}.

\\Forschung
auf dem Gebiet der AR in der Qualitätskontrolle und Inspektion hat gezeigt,
dass die Nutzung dieser Technologie zu einer höheren Inspektionsgenauigkeit,
einer schnelleren Fehlererkennung und einer verbesserten Effizienz führen kann.
Die Integration von AR in diesen Bereich bietet somit großes Potenzial, die
Qualitätssicherung in verschiedenen Industriezweigen zu optimieren und die
Inspektionsprozesse zu verbessern.\cite{etonam2019augmented}
\section{Forschungsbereiche}
Die Forschung im Bereich der Augmented Reality (AR) umfasst eine Vielzahl von
Disziplinen und Forschungsbereichen, die darauf abzielen, die Technologie,
Anwendungen und Interaktionsmöglichkeiten von AR weiterzuentwickeln. Einer
dieser Forschungsbereiche ist die Darstellung und Visualisierung von
AR-Inhalten. Hierbei werden innovative Methoden zur realistischen Integration
virtueller Objekte in die reale Umgebung untersucht, einschließlich der
Verbesserung von Beleuchtung, Schattenwurf und Perspektive, um eine nahtlose
und überzeugende AR-Erfahrung zu schaffen. Ein weiterer Forschungsschwerpunkt
liegt auf der Erfassung und Verarbeitung von Umgebungsdaten, um präzise
Tracking- und Lokalisierungstechniken zu entwickeln, die es AR-Systemen
ermöglichen, die Position und Ausrichtung von Benutzern und Objekten in
Echtzeit zu verfolgen. Darüber hinaus wird in der AR-Forschung intensiv an der
Entwicklung neuer Interaktionsmethoden und Benutzerschnittstellen gearbeitet,
um die intuitive und immersive Interaktion mit AR-Inhalten zu ermöglichen. Dies
umfasst die Untersuchung von Gestensteuerung, Sprachbefehlen, haptischer
Rückmeldung und anderen innovativen Eingabemethoden. Ein weiteres wichtiges
Forschungsgebiet ist die Entwicklung von AR-Anwendungen und -Anwendungsfällen
in verschiedenen Bereichen wie Bildung, Medizin, Unterhaltung, Architektur und
Industrie. Hierbei werden neue Einsatzmöglichkeiten von AR erforscht und
prototypische Anwendungen entwickelt, die das Potenzial der Technologie
demonstrieren. Die AR-Forschung ist ein dynamisches Feld, das ständig erweitert
wird, um neue Herausforderungen anzugehen und die Nutzung von AR in
verschiedenen Domänen zu optimieren.
\section{Potenzial von AR für neue Geschäftsmodelle}

Neben den bestehenden Anwendungsfällen von AR in der Industrie ergeben sich
durch den stetigen technologischen Fortschritt neue potentielle Anwendungsfälle
und Geschäftsmodelle. Der Tabelle 1 sind die häufigsten Suchanfragen im
Zusammenhang mit Augmented Reality zu entnehmen. Abbildung
\ref{fig:studienverteilung} veranschaulicht dabei die Anzahl an Primärstudien
in den verschiedenen Bereichen von Ar. Dies kann einen guten Eindruck über
aktuelle Trends in diesem Bereich geben. Im folgenden Kapitel werden einige der
interessantesten Potenziale von AR in der Industrie dargestellt.\\

\begin{table}[h]
    \centering
    \captionsetup{font=small}
    \label{tabelle1}
    \renewcommand{\arraystretch}{1.35} % Anpassung des Zeilenabstands um den Faktor 1.2

    \begin{tabular}{c|l}
        \multicolumn{1}{c|}{Platzierung} & \multicolumn{1}{c}{\centering Suchanfrage}  \\
        \hline
        1                                & ''Augmented Reality'' AND ''Manufacturing'' \\
        2                                & ''Augmented Reality'' AND ''Production"''   \\
        3                                & ''Augmented Reality'' AND ''Assembly''      \\
        4                                & ''Augmented Reality'' AND ''Shop floor''    \\
        5                                & ''Augmented Reality'' AND ''Factory floor'' \\
        \hline

    \end{tabular}
    \caption{Suchbegriffe im Bezug auf AR in der Industrie \cite{de2018augmented}}

\end{table}

\begin{figure}[h]
    \centering
    \includesvg[width=0.9\columnwidth]{bilder/svg/studienverteilung_ar.svg}
    \caption[width=0.7schaff\columnwidth]{Anzahl der Primärstudien pro potenziellem Einsatzbereich von Augmented Reality in der Industrie \cite{de2020survey}}
    \label{fig:studienverteilung}
\end{figure}

\subsection{Produktdesign und -entwicklung}
Augmented Reality (AR) bietet ein großes Potenzial für den Anwendungsbereich
des Produktdesigns. Durch die Integration von AR-Technologien in den
Designprozess können Designer und Ingenieure hochwertige Vorschauen und
interaktive Simulationen ihrer Produkte erstellen. Die Nutzung einer
cloud-basierten Anwendung ermöglicht Echtzeit-Kollaboration und Meetings
zwischen Designern weltweit, unabhängig von ihrem Standort. Dies fördert die
Zusammenarbeit, verbessert die Kommunikation und ermöglicht präzisere
Entscheidungen während des Designprozesses. Mit AR können virtuelle Prototypen
in der realen Welt platziert werden, um das Aussehen, die Funktionalität und
die Benutzererfahrung zu bewerten. Dies hilft dabei, Designkonflikte und
-probleme frühzeitig zu erkennen und zu lösen. Darüber hinaus ermöglicht AR
eine immersive Darstellung von Produkten, sodass Kunden und Stakeholder sie vor
dem eigentlichen Bau oder der Produktion in einem realistischen Kontext erleben
können. Dies verbessert nicht nur das Verständnis des Produkts, sondern
ermöglicht auch wertvolles Feedback und iteratives Design. Durch die Nutzung
von AR im Produktdesign können Unternehmen die Effizienz steigern, die
Fehlerquote verringern und letztendlich bessere Produkte entwickeln, die den
Bedürfnissen der Kunden entsprechen.\cite{mourtzis2020augmented}

\subsection{Fernwartung und Montage}
Die Nutzung von Augmented Reality (AR) zur Unterstützung von Wartungsarbeiten
an industriellen Geräten hat sich als vielversprechender Anwendungsbereich
erwiesen. Durch die Integration von AR in die Fernwartung könnten sich sogar
noch effektivere Möglichkeiten eröffnen. Bereits erste funktionierende Ansätze
zeigen, dass AR eine bedeutende Rolle bei der Durchführung von
Fernwartungsarbeiten spielen kann. \cite{masoni2017supporting} \\Ein weiterer
Schritt in dieser Entwicklung wäre der verstärkte Einsatz von AR in der
Produktmontage. Im Vergleich zu gedruckten Handbüchern bietet AR klare
Vorteile, da sie die Aufmerksamkeit auf das zu bearbeitende Objekt lenkt und
Fehler erkennt und reduziert. Frühere Arbeiten auf diesem Gebiet waren
experimentell und konzentrierten sich auf spezifische Objekte. Dabei wurden
häufig Tracking-Techniken auf Basis von Referenzmarkierungen verwendet. Durch
Fortschritte in Bereichen wie Deep Learning kann AR jedoch zunehmend komplexe
Montageprozesse unterstützen. Dennoch ist es teilweise schwierig, die konkreten
Vorteile von AR in diesem Bereich nachzuweisen, Studien in diesem Bereich
kommen zu unterschiedlichen Ergebnissen \cite{tang2003comparative}. Dies kann
auf verschiedene Faktoren zurückzuführen sein, wie z.B. die Komplexität der
Montageprozesse, die Qualität der eingesetzten AR-Anwendungen und die
Einarbeitung der Mitarbeiter in die Nutzung der Technologie. Trotzdem wird die
kontinuierliche technologische Weiterentwicklung AR in Zukunft noch
leistungsfähiger machen und den Unternehmen noch mehr Möglichkeiten bieten,
komplexe Montageprozesse zu unterstützen und zu optimieren. Es ist zu erwarten,
dass AR in der Industrie einen immer größeren Stellenwert einnehmen wird und zu
einer effektiven Unterstützung bei der Montage von Produkten
wird.\cite{8951930}

\subsection{Marketing}
Die Potenziale von Augmented Reality (AR) im Bereich des Marketings sind
vielfältig und vielversprechend. AR ermöglicht es Unternehmen, interaktive und
immersivere Markenerlebnisse für ihre Zielgruppe zu schaffen. Durch die
Einbindung digitaler Inhalte in die reale Welt können Produkte und
Dienstleistungen auf innovative und ansprechende Weise präsentiert werden. AR
bietet die Möglichkeit, Kunden in ihren individuellen Entscheidungskontexten
anzusprechen und personalisierte, relevante Informationen bereitzustellen.
Darüber hinaus eröffnet AR neue Wege der Kundeninteraktion, indem es
spielerische Elemente integriert. Durch die Schaffung von emotionalen Bindungen
und positiven Markenerlebnissen kann AR das Kundenengagement und die
Markenloyalität erhöhen. Unternehmen haben auch die Möglichkeit, AR als
Instrument für datengesteuertes Marketing einzusetzen, indem sie Nutzungsdaten
und Interaktionsmuster erfassen und analysieren, um gezielte Marketingmaßnahmen
abzuleiten. Die kontinuierlichen Fortschritte in der AR-Technologie und die
steigende Verbreitung von AR-fähigen Endgeräten eröffnen spannende Perspektiven
für innovative Marketingstrategien und -kampagnen, die das Kundenerlebnis
revolutionieren können.\cite{chylinski2020augmented,rauschnabel2019augmented}

\subsection{Schulung und Training}
Augmented Reality (AR) wird immer wieder im Zusammenhang mit Schulungsprozessen
in der Industrie erwähnt. Besonders interessant ist dabei der Einsatz von
AR-basierten Assistenzsystemen während des Schulungsprozesses. Theoretisch
könnten dadurch der Umgang mit potenziell gefährlichen Situationen in einer
sicheren Umgebung trainiert werden. Auch das Erlernen von Montageaufgaben ist
mit AR-Unterstützung möglich. Dabei können Anleitungen in Papierform durch AR
ersetzt und Fehler beim Aufbau durch den Monteur erkannt und mitgeteilt werden.
Allerdings gibt es in diesem Bereich keine klaren Studien, die den besseren
Lernerfolg durch AR belegen. Am besten schneidet weiterhin die persönliche
Schulung ab. Das AR-Assistenzsystem verhindert effektiv ein fehlerhaftes
Erlernen von Inhalten, erreicht jedoch nicht die Geschwindigkeit der
persönlichen Schulung. Im Vergleich zu einer individuellen Schulung mit einem
gedruckten Handbuch konnte keine höhere Effizienz der AR-Schulung nachgewiesen
werden. Bezüglich der langfristigen Wissensnachhaltigkeit zeigt sich, dass es
keine Unterschiede im Erinnerungsvermögen der Teilnehmer an den Montageprozess
gab, unabhängig von der Schulungsmethode. Die Ergebnisse legen nahe, dass
AR-Assistenzsysteme als hilfreiche Werkzeuge für die Schulung von Arbeitern in
Montageaufgaben eingesetzt werden können, insbesondere um fehlerhaftes Lernen
zu vermeiden. Es ist jedoch zu beachten, dass weitere Vorteile wie eine höhere
Schulungseffizienz in dieser Studie nicht belegt werden konnten. Dennoch ist zu
erwarten, dass die Effizienz der Schulungen durch den technischen Fortschritt
im Bereich AR weiter zunehmen wird und AR-gestützte Schulungen Unternehmen
einen großen Vorteil bieten können.
\section{Fazit}
Der Einsatz von Augmented Reality (AR) in der Industrie bietet immense
Potenziale für die Optimierung von Arbeitsabläufen, Schulungen,
Qualitätskontrolle und vielem mehr. AR ermöglicht es Mitarbeitern, relevante
Informationen in Echtzeit einzusehen und interaktiv mit virtuellen Inhalten zu
interagieren, was zu einer Steigerung der Effizienz, Produktivität und
Fehlerminimierung führt. Durch die Integration von AR in bestehende
Arbeitsprozesse können Unternehmen ihre Wettbewerbsfähigkeit stärken und neue
Geschäftsmöglichkeiten erschließen.

Allerdings gibt es auch Herausforderungen zu bewältigen, wie die nahtlose
Integration von AR in bestehende Systeme, die Sicherheit sensibler
Unternehmensdaten und die Wartung von AR-Hardware und -Software. Durch eine
sorgfältige Planung, Zusammenarbeit zwischen den beteiligten Stakeholdern und
den Einsatz geeigneter Lösungsansätze können diese Herausforderungen
erfolgreich bewältigt werden.

Ein Zukunftsausblick zeigt, dass AR in der Industrie noch weiteres Potenzial
hat. Die Integration von Künstlicher Intelligenz (KI) und maschinellem Lernen
eröffnet neue Möglichkeiten für automatische Erkennung, Analyse und
Personalisierung von AR-Erlebnissen. Mit kontinuierlicher Forschung und
Entwicklung sowie enger Zusammenarbeit zwischen Industrie und Wissenschaft
können die Potenziale von AR weiter erschlossen und innovative
Anwendungsszenarien entwickelt werden.

Insgesamt lässt sich sagen, dass AR in der Industrie eine vielversprechende
Technologie ist, die einen positiven Einfluss auf die Arbeitswelt haben kann.
Unternehmen sollten AR als strategischen Ansatz betrachten, um ihre Prozesse zu
verbessern, die Mitarbeiterleistung zu steigern und ihre Wettbewerbsfähigkeit
zu stärken. Die erfolgreiche Integration von AR erfordert jedoch eine
sorgfältige Planung, enge Zusammenarbeit und kontinuierliche Anpassung an die
spezifischen Anforderungen der Industrie. Durch die richtige Herangehensweise
können Unternehmen die Vorteile von AR nutzen und einen Mehrwert für ihr
Geschäft schaffen.
\printbibliography
\end{document}