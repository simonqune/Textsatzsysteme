\section{Bestehende Herausforderungen und Forschungsansätze}
Trotz der enormen Fortschritte im Bereich der Augmented Reality in den letzten
Jahren, bestehen weiterhin einige Herausforderungen, die es zu bewältigen gilt.
In diesem Zusammenhang gibt es viele Forschungsansätze im Bereich der Augmented
Reality.
\subsection{Tracking}
Eine präzise und zuverlässige Tracking-Methode stellt eine essenzielle
Komponente für AR dar. Obwohl in den letzten Jahren erhebliche Fortschritte in
diesem Bereich erzielt wurden, sind die Verfahren noch nicht vollständig
ausgereift. Die derzeitigen technologischen Möglichkeiten erlauben eine genaue
Bestimmung der Position des AR-Systems ohne Verwendung von Markern durch
Bildverarbeitungstechniken und Deep-Learning. Dies erfolgt beispielsweise durch
den Abgleich von hinterlegten Objektmodellen mit der realen
Umgebung.\cite{7907444} Allerdings sind in Produktionsumgebungen oft nicht
ausreichend markanten Stellen vorhanden, die ein zuverlässiges Tracking
ermöglichen \cite{devagiri2022augmented}. Darüber hinaus werden häufig nur
wenige Punkte in der Umgebung zur Lokalisierung genutzt, was zu einer teilweise
unpräzisen Erfassung der Umgebung führen kann \cite{chen2019design}.

\subsection{Herausforderungen im Bereich der AR-Darstellung}

Im industriellen Sektor kommen verschiedene AR-Darstellungstechniken, am
häufigste allerdings Head Mounted Displays (HMDs) zum Einsatz, wie in Abbildung
\ref{fig:PercDistributionARDisplay} zu erkennen ist. Bei diesen gibt es nach
wie vor einige Herausforderungen, die angegangen werden müssen. Oftmals ist das
Sichtfeld, das durch das AR-System bereitgestellt wird, noch zu klein, um ein
vollständig immersives Gefühl zu erzeugen. Zudem kann der Rechenaufwand je nach
Auflösung und Sichtbereich zu Verzögerungen führen, wodurch Einblendungen in
die reale Welt nicht in Echtzeit erfolgen können. Auch zu langsame
Bildwiederholraten können das Gefühl der Immersion beeinträchtigen. Darüber
hinaus ist die Darstellung von Schatten und Lichteffekten sehr rechenintensiv
und kann häufig nicht in Echtzeit erfolgen \cite{zhao2020pointar}. Um die
genannten Probleme anzugehen, werden kontinuierlich Forschungen betrieben, um
effizientere Hardware- und Softwarelösungen zu entwickeln. Insbesondere im
Bereich der Akkutechnik sind weitere Fortschritte erforderlich, um das Gewicht
von AR-Brillen zu reduzieren und die Batterielaufzeit zu verbessern, um deren
praktischen Einsatz zu optimieren \cite{arena2022overview}.

Semi-Transparente-Displays haben zusätzliche Nachteile, wie beispielsweise eine
beschränkte Sichtfeldunterstützung und keine vollständige Verdeckung der realen
Welt. Zudem können die AR-Einblendungen aufgrund des starken Umgebungslichts
teilweise schwer erkennbar sein \cite{itoh2021towards}. Um diese Mängel zu
beheben, wurden verschiedene Designs für durchsichtige Displays entwickelt. Es
wurde beispielsweise untersucht, wie elektronische Maskierungselemente zu
durchsichtigen Displays hinzugefügt werden können, um die Problematik einer
besseren Überblendung zu lösen \cite{8676153}.

Eine weitere vielversprechende Forschungsrichtung im Bereich der Displays liegt
in der Entwicklung von Head-Mounted Projection Displays (HMPD). Frühe
Projektor-Systeme waren sperrig und nicht tragbar, jedoch konnten durch
Fortschritte in der Pico-Projektortechnologie diese Einschränkungen überwunden
werden. Es wird auch daran geforscht, wie die Interaktion mit projizierten
AR-Inhalten verbessert werden kann, beispielsweise durch Gestensteuerung oder
den Einsatz physischer Objekte als Benutzerschnittstelle
\cite{hartmann2020aar}.

Ein weiterer Ansatz sind kontaktlinsenbasierte Displays. Das Ziel besteht
darin, ein Head-Mounted Display zu entwickeln, das für andere Personen um den
Benutzer herum nicht wahrnehmbar ist. Durch den Einsatz von MEMS-Technologie
und drahtloser Energie- und Datenübertragung könnten aktive Pixel in eine
Kontaktlinse integriert werden. Es wurden bereits Prototypen entwickelt, jedoch
gibt es noch Herausforderungen wie die Integration von Optiken, sowie die
kontinuierliche Stromversorgung und Datenübertragung, die bewältigt werden
müssen \cite{chen2019design}.

\begin{figure}[h]
    \centering
    \includesvg[width=1\columnwidth]{bilder/svg/distribution_AR_display.svg}
    \caption[width=0.9\columnwidth]{Prozentuale Verteilung der AR-Darstellungstechniken in der Industrie \cite{de2020survey} }
    \label{fig:PercDistributionARDisplay}
\end{figure}

\subsection{Datenschutz}

Bei tragbaren AR-Systemen bestehen häufig Bedenken hinsichtlich des
Datenschutzes. Da AR zunehmend in industrieller Fertigung, Wartung und Schulung
eingesetzt wird, können Informationen über Maschinen, Produktionsprozesse und
geschützte Betriebsgeheimnisse visualisiert werden. Datenschutzprobleme können
auftreten, wenn AR-Anwendungen Zugriff auf sensible Unternehmensdaten erhalten
oder unbefugt Informationen über Produktionsabläufe und -muster sammeln.
Unternehmen müssen sicherstellen, dass die Daten, die in AR-Anwendungen
verwendet werden, angemessen geschützt und vor unbefugtem Zugriff gesichert
sind. Darüber hinaus können AR-Brillen oder -Geräte in industriellen Umgebungen
auch die Privatsphäre der Mitarbeiter beeinträchtigen, insbesondere wenn sie
kontinuierlich Video- oder Audioaufnahmen machen. Es ist wichtig, klare
Richtlinien und Vereinbarungen zum Schutz der Privatsphäre der Mitarbeiter
aufzustellen und sicherzustellen, dass AR-Technologien verantwortungsvoll
eingesetzt werden, um sowohl den Datenschutz als auch die
Arbeitsplatzsicherheit zu gewährleisten.\cite{de2018augmented,9613426}