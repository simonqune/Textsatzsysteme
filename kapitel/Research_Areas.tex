\section{Forschungsbereiche}
Die Forschung im Bereich der Augmented Reality (AR) umfasst eine Vielzahl von
Disziplinen und Forschungsbereichen, die darauf abzielen, die Technologie,
Anwendungen und Interaktionsmöglichkeiten von AR weiterzuentwickeln. Einer
dieser Forschungsbereiche ist die Darstellung und Visualisierung von
AR-Inhalten. Hierbei werden innovative Methoden zur realistischen Integration
virtueller Objekte in die reale Umgebung untersucht, einschließlich der
Verbesserung von Beleuchtung, Schattenwurf und Perspektive, um eine nahtlose
und überzeugende AR-Erfahrung zu schaffen. Ein weiterer Forschungsschwerpunkt
liegt auf der Erfassung und Verarbeitung von Umgebungsdaten, um präzise
Tracking- und Lokalisierungstechniken zu entwickeln, die es AR-Systemen
ermöglichen, die Position und Ausrichtung von Benutzern und Objekten in
Echtzeit zu verfolgen. Darüber hinaus wird in der AR-Forschung intensiv an der
Entwicklung neuer Interaktionsmethoden und Benutzerschnittstellen gearbeitet,
um die intuitive und immersive Interaktion mit AR-Inhalten zu ermöglichen. Dies
umfasst die Untersuchung von Gestensteuerung, Sprachbefehlen, haptischer
Rückmeldung und anderen innovativen Eingabemethoden. Ein weiteres wichtiges
Forschungsgebiet ist die Entwicklung von AR-Anwendungen und -Anwendungsfällen
in verschiedenen Bereichen wie Bildung, Medizin, Unterhaltung, Architektur und
Industrie. Hierbei werden neue Einsatzmöglichkeiten von AR erforscht und
prototypische Anwendungen entwickelt, die das Potenzial der Technologie
demonstrieren. Die AR-Forschung ist ein dynamisches Feld, das ständig erweitert
wird, um neue Herausforderungen anzugehen und die Nutzung von AR in
verschiedenen Domänen zu optimieren.