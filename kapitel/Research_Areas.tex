\section{Forschungsbereiche}
Die Forschung im Bereich der Augmented Reality (AR) umfasst eine Vielzahl von
Disziplinen und Forschungsbereichen, die darauf abzielen, die Technologie,
Anwendungen und Interaktionsmöglichkeiten von AR weiterzuentwickeln.

\subsection{Tracking}
Tracking ist ein wichtiger Bereich der AR-Forschung, da es eine der zentralen
Technologien für die Umsetzung von AR-Erfahrungen ist. Es wurden verschiedene
Tracking-Systeme entwickelt, angefangen von einfachen Marker-basierten Systemen
bis hin zu natürlichen Merkmalen und hybriden Sensormethoden. Dennoch sind
weitere Fortschritte erforderlich, um das Ziel einer umfassenden ''Anywhere
Augmentation'' zu erreichen, bei der Nutzer eine überzeugende AR-Erfahrung in
jeder Umgebung haben können.

Für die zuverlässige Outdoor-Augmented-Reality sind Tracking-Methoden
erforderlich, die eine genaue Standortbestimmung über große Flächen
ermöglichen. GPS kann zur Positionsbestimmung verwendet werden, liefert jedoch
in mobilen Geräten der Verbraucher nur eine durchschnittliche Genauigkeit von
5-10 Metern. Eine alternative Methode besteht darin, computergestützte
Bildverarbeitungstechniken in Verbindung mit GPS und inertialen Sensoren zu
verwenden, um die Kameraposition relativ zu bekannten visuellen Merkmalen
abzuschätzen. Diese Methode ist jedoch schwierig auf großflächiges Tracking
anwendbar.

Ein vielversprechender Ansatz für weitreichendes Tracking in unvorbereiteter
Außenumgebung ist die Kombination von Panoramabildern zur Erstellung eines
Punktewolkenmodells. Dieses Modell kann für die positionsbasierte Lokalisierung
über einen Remote-Server und für die Echtzeitverfolgung auf einem mobilen Gerät
verwendet werden. Die Genauigkeit der Verfolgung beträgt weniger als 25 cm
Fehler bei der Positionsbestimmung und unter 0,5 Grad Fehler bei der Rotation.

Für das präzise Indoor-Tracking werden verschiedene Methoden untersucht, wie
Ultraschall, Kameras mit Laser-Trackern, Marker-basiertes Tracking und hybride
Systeme mit Computer Vision, inertialen Sensoren und Ultra-Wideband-Tracking.
Ein vielversprechender Forschungsansatz ist die Verwendung von
Handheld-Tiefensensoren wie dem Google Tango-Projekt. Dies ermöglicht präzise
Innenraumpositionierung und bietet eine ideale Plattform für Indoor-AR.

Eine Herausforderung besteht darin, nahtlos zwischen Outdoor- und
Indoor-Tracking-Umgebungen zu wechseln. Bisher wurden Lösungen entwickelt, die
GPS mit Marker-basiertem Tracking kombinieren oder drahtlose Netzwerke mit GPS
verbinden. Weitere Forschung ist erforderlich, um das Ziel eines
kontinuierlichen und ubiquitären Trackingsystems zu erreichen, das nahtlos
zwischen verschiedenen Umgebungen funktioniert.
\subsection{Interaktion}

In Abschnitt 7 wird ein Überblick über verschiedene Interaktionsmethoden für
Augmented Reality gegeben. Frühe AR-Schnittstellen verwendeten Techniken, die
von Desktop-Schnittstellen oder Virtual Reality inspiriert waren. Im Laufe der
Zeit wurden jedoch innovativere Methoden wie Tangible AR oder natürliche
Gesteninteraktion eingesetzt. Es besteht immer noch erheblicher
Forschungsbedarf in neuen Interaktionsmethoden, insbesondere in den Bereichen
intelligente Systeme, hybride Benutzerschnittstellen und kollaborative Systeme.
In diesem Abschnitt werden einige Möglichkeiten in jedem dieser Bereiche kurz
vorgestellt.

Bisherige Forschungsergebnisse haben gezeigt, dass AR eine sehr natürliche Art
der Interaktion mit virtuellen Inhalten ist, aber in vielen Fällen waren die
Schnittstellen selbst nicht sehr intelligent und reagierten nicht
unterschiedlich auf Benutzereingaben. Das Feld der Intelligent User Interfaces
(IUI) hat sich in den letzten zwanzig Jahren entwickelt, in dem untersucht
wird, wie künstliche Intelligenz mit Methoden der Mensch-Computer-Interaktion
kombiniert werden kann, um reaktionsschnellere Schnittstellen zu erzeugen. Es
gibt jedoch nur wenig Forschung zu IUI-Methoden in AR. Einige Forscher haben
begonnen, den Einsatz von virtuellen Charakteren zu erkunden, die begrenzte
Intelligenz zeigen. Zum Beispiel wurde das Welbo-Interface entwickelt, bei dem
ein Charakter in der realen Welt zu sehen war und auf einfache Sprachbefehle
reagierte. Ein anderes Beispiel ist Mr Virtuoso, eine AR-Schnittstelle, die
einen virtuellen Charakter zur Vermittlung von Kunstwissen einsetzte.

Ein vielversprechender Forschungsbereich ist die intelligente Trainingssysteme
(ITS). Frühere Forschungen haben gezeigt, dass sowohl AR als auch
ITS-Anwendungen das Training erheblich verbessern können. Beispielsweise
ermöglicht AR-Technologie das Überlagern virtueller Hinweise auf die Ausrüstung
von Arbeitern und hilft bei räumlichen Aufgaben. ITS-Anwendungen ermöglichen es
den Menschen, eine auf ihren individuellen Lernstil zugeschnittene
Lernerfahrung zu haben, indem sie intelligente, reaktionsschnelle Rückmeldungen
bieten. Es wurde gezeigt, dass ITS die Lernerfolge um mindestens eine Note
verbessern, das Lernen erheblich beschleunigen und beeindruckende Ergebnisse
bei der Wissensübertragung erzielen können. Es gibt jedoch nur wenig Forschung,
die untersucht, wie beide Technologien kombiniert werden können.

Ein weiterer vielversprechender Forschungsbereich ist die Entwicklung hybrider
AR-Schnittstellen und -Interaktionen. Anfangs waren AR-Systeme eigenständige
Anwendungen, bei denen der Benutzer sich ausschließlich auf die
AR-Schnittstelle und die Interaktion mit den virtuellen Inhalten konzentrierte.
Mit der Zeit wurden jedoch hybride Schnittstellen entwickelt, die AR mit
anderen Interaktionsmethoden kombinieren. Es gibt interessante Möglichkeiten
für Forschung in den Bereichen AR und Ubiquitous Computing, AR und VR sowie AR
und herkömmliche Desktop-Schnittstellen.

Eine weitere Möglichkeit besteht darin, AR mit intelligenten
Benutzerschnittstellen, VR-Schnittstellen und Ubiquitous
Computing-Schnittstellen zu kombinieren. Dies könnte dazu führen, dass VR und
AR nahtlos ineinander übergehen und mit Ubiquitous Computing-Technologien
verschmelzen. Diese hybriden Schnittstellen könnten den Benutzern ständigen
Zugriff auf Informationen ermöglichen und verschiedene
Schnittstellendarstellungen verwenden.

Insgesamt sind die aktuellen AR-Trainingssysteme nicht intelligent, und
aktuelle ITS verwenden keine AR-Schnittstelle für ihre Benutzerschnittstelle.
Es besteht die Notwendigkeit, intelligente AR-Trainingssysteme zu entwickeln,
die Benutzern Feedback zur Qualität ihrer Aufgaben geben. Die Kombination von
AR mit IUI-Methoden und anderen Schnittstellentechnologien eröffnet
interessante Möglichkeiten für weiterführende Forschung.
\subsection{Displays}
Im Bereich der AR-Displaytechnologie hat es seit Ivan Sutherlands erstem System
erhebliche Fortschritte gegeben, aber aktuelle Displays sind immer noch weit
von Sutherlands Vision des "Ultimate Display" entfernt. Es gibt wichtige
Forschungsmöglichkeiten in der Gestaltung von Head-Mounted Displays,
Projektionstechnologie, Kontaktlinsen-Displays und anderen Bereichen.

Traditionelle optische Durchsicht-Displays haben jedoch einige Nachteile, wie
zum Beispiel eine beschränkte Sichtfeldunterstützung und keine echte Verdeckung
der realen Welt. Die ideale Anzeige wäre eine, die ein großes Sichtfeld
unterstützt, Verdeckung der realen Welt ermöglicht und Bilder auf verschiedenen
Fokusebenen liefert, und das alles in einem kleinen und unauffälligen
Formfaktor. Forscher haben in jedem dieser Bereiche Fortschritte gemacht, aber
es gibt noch viel Arbeit zu tun.

Es wurden verschiedene Designs für optische Durchsicht-Displays entwickelt, um
diese Mängel zu beheben. Zum Beispiel haben Kiyokawa et al. untersucht, wie
elektronische Maskierungselemente zu optischen Durchsicht-Displays hinzugefügt
werden können, um die Verdeckungsproblematik zu lösen. Andere Forscher haben
sich mit der Erweiterung des Sichtfeldes und der Variation der Fokusebene
befasst. Es gab auch Versuche, all diese Probleme in einem Design zu lösen.

Eine weitere interessante Forschungsmöglichkeit im Bereich Displays liegt in
der Entwicklung von Head-Mounted Projection Displays (HMPD). Frühe
Projektor-Systeme waren sperrig und nicht tragbar, aber Fortschritte in der
Pico-Projektortechnologie haben diese Einschränkungen überwunden. Es gibt auch
Forschung darüber, wie die Interaktion mit projizierten AR-Inhalten verbessert
werden kann, zum Beispiel durch Gestensteuerung oder den Einsatz physischer
Objekte als Benutzerschnittstelle.

Ein vielversprechender Ansatz sind kontaktlinsenbasierte Displays. Das Ziel ist
es, ein Head-Mounted Display zu entwickeln, das für andere Personen um den
Benutzer herum nicht wahrnehmbar ist. Durch den Einsatz von MEMS-Technologie
und drahtloser Energie- und Datenübertragung könnten aktive Pixel in eine
Kontaktlinse integriert werden. Es wurden bereits Prototypen entwickelt, aber
es gibt noch Herausforderungen wie die Integration von Optiken, ausreichende
Sauerstoffversorgung der Hornhaut und kontinuierliche Stromversorgung und
Datenübertragung, die gelöst werden müssen.
\subsection{Soziale Akzeptanz}

Soziale Akzeptanz ist ein wichtiger Faktor, der die Verbreitung von Augmented
Reality (AR) beeinflusst, insbesondere bei tragbaren oder mobilen Systemen.
Obwohl AR-Systeme in Bezug auf Größe und Gewicht immer kleiner geworden sind,
gibt es immer noch erheblichen Widerstand in der Gesellschaft gegenüber Geräten
wie Google Glass. Eine Umfrage in den USA ergab beispielsweise, dass nur 12
Prozent der Befragten bereit wären, "Augmented-Reality-Brillen" von einer
Marke, der sie vertrauen, zu tragen. Die Gründe für diese Zurückhaltung können
Datenschutzbedenken, die Angst, lächerlich auszusehen, oder die Sorge, zum Ziel
von Dieben zu werden, sein.

Diese Bedenken beschränken sich nicht nur auf tragbare AR-Systeme. Wenn eine
Person beispielsweise mit einem mobilen Telefon oder Tablet durch eine Stadt
geht und eine AR-Browser-Anwendung verwendet, um sich zurechtzufinden oder
AR-Inhalte anzuzeigen, muss sie das Telefon auf Augenhöhe vor sich halten,
während sie geht. Diese unnatürliche Haltung kann sich albern anfühlen und
andere Menschen denken lassen, dass sie gefilmt werden.

Obwohl einige Forscher die soziale Akzeptanz als wichtigen Aspekt von AR
hervorgehoben haben, gab es zunächst wenig Forschung zu diesem Thema. In
jüngerer Zeit wurden jedoch einige Arbeiten durchgeführt. Zum Beispiel wurden
positive Ergebnisse erzielt, als AR-Technologie als Instrument für klinische
Schulungen in einem Krankenhaus eingesetzt wurde. Auch in einer universitären
Umgebung zeigten Studien, dass die Mehrheit der Studenten AR für das Lehren und
Lernen als nützlich erachtete.

Es gibt jedoch nur wenige Untersuchungen zur sozialen Akzeptanz von AR in
öffentlichen oder sozialen Situationen. Es besteht ein Bedarf an weiterer
Forschung auf diesem Gebiet, insbesondere im Hinblick auf die Erfahrung von
Benutzern mit mobilen AR-Diensten und den damit verbundenen sozialen und
emotionalen Aspekten. Bisherige Studien zeigen, dass soziale Akzeptanzprobleme
bei Personen, die ständig eine AR-Brille tragen, wahrscheinlich höher sind als
bei kurzzeitiger Nutzung von mobilen oder tragbaren AR-Systemen.

Es ist zu erwarten, dass sich die soziale Akzeptanz verbessert, wenn
AR-Technologie unauffälliger wird und Displays sowie Eingabegeräte in Kleidung
integriert werden. Die Erforschung der sozialen Akzeptanz in tragbaren oder
mobilen AR-Erlebnissen, insbesondere in öffentlichen Umgebungen, ist ein
wichtiger Bereich für zukünftige Arbeit in der AR-Gemeinschaft.