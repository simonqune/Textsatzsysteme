\section{Einleitung}
MOOOOOOOOOOOOOOOOOOOOOOOOOOOOOOOOOOOOOOOOOIIIIIIIIIIIIIIIIIIIIIIIIIIIIIIIIIIIIIIIIIIIIIIIIIIIIIIIIIIIIIIIIIIIIIIIN
Augmented Reality (AR) hat in den letzten Jahren zunehmend an Bedeutung
gewonnen und wird auch in der Industrie verstärkt eingesetzt. AR ermöglicht es,
digitale Inhalte in die physische Welt zu projizieren und so die Realität mit
virtuellen Informationen und Objekten zu erweitern. Dies eröffnet interessante
Möglichkeiten für die Optimierung von Prozessen, Steigerung der Produktivität
und die Entwicklung innovativer Geschäftsmodelle in der Industrie.

AR in der Industrie: Stand der Technik AR findet bereits vielfältige
Anwendungen in der Industrie. Es wird für Training und Schulung von
Mitarbeitern, Wartung und Instandhaltung von Maschinen und Anlagen,
Produktentwicklung und Optimierung von Produktionsprozessen eingesetzt. Dabei
kommen unterschiedliche AR-Technologien und Plattformen wie Head-mounted
Displays (HMDs), Smart Glasses oder Marker-basierte AR-Systeme zum Einsatz.
