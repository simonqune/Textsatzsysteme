\section{AR in der Industrie: Stand der Technik}

Der aktuelle Stand der Technik in Bezug auf Augmented Reality in der Industrie
zeigt ein stetiges Wachstum und eine kontinuierliche Weiterentwicklung. Es gibt
bereits zahlreiche Anwendungen und Lösungen, die in verschiedenen
Industriezweigen erfolgreich eingesetzt werden. Ein Bereich, in dem AR bereits
Anwendung findet, ist die Wartung und Instandhaltung von Maschinen und Anlagen.
Techniker können mithilfe von AR-Brillen oder anderen AR-Geräten in Echtzeit
Anleitungen und Informationen erhalten, um Reparaturen durchzuführen oder
Wartungsprozesse zu optimieren. Auch in der Fertigungsindustrie wird AR
eingesetzt, um Arbeitskräfte bei der Montage von Produkten zu unterstützen.
AR-Anwendungen können visuelle Anweisungen bereitstellen, um die Genauigkeit
und Effizienz von Montageprozessen zu verbessern. Darüber hinaus wird Augmented
Reality auch für Schulungs- und Trainingsszenarien in der Industrie eingesetzt.
Mitarbeiter können in virtuellen Umgebungen geschult werden, um komplexe
Aufgaben zu erlernen oder gefährliche Situationen zu simulieren, ohne
tatsächlich physisch anwesend zu sein. Auch in der Qualitätssicherung findet AR
Anwendung, um Fehler zu minimieren und die Produktqualität zu verbessern.
AR-Anwendungen können visuelle Prüfungen durchführen und Mitarbeiter bei der
Identifikation von Mängeln unterstützen. Weitere Anwendungsfelder von AR in der
Industrie sind die Logistik und Lagerverwaltung, wo AR bei der Navigation,
Kommissionierung und Verfolgung von Waren eingesetzt wird. Insgesamt zeigt der
Stand der Technik, dass Augmented Reality bereits vielfältige Anwendungen in
der Industrie hat und weiterhin Potenzial für zukünftige Entwicklungen bietet.