\section{Bestehende Einsatzgebiete von Augmented Reality in der Industrie}

AR hat sich in den letzten Jahren zu einer vielversprechenden Technologie in
verschiedenen Branchen entwickelt, insbesondere in der Industrie. Unternehmen
setzen AR erfolgreich ein, um die Effizienz zu steigern, Fehler zu reduzieren
und die Sicherheit am Arbeitsplatz zu verbessern. Dabei findet AR besonders in
den nachfolgenden Bereichen Anwendung.

\subsection{Wartung und Instandhaltung}
Die Integration von AR in den Bereich der Wartung und Instandhaltung hat das
Potenzial, die Effizienz und Genauigkeit dieser Prozesse signifikant zu
verbessern \cite{liu2022probing}. Durch die Nutzung von AR-Brillen oder anderen
AR-Geräten können Techniker während ihrer Arbeit visuelle Informationen und
Anweisungen direkt in ihr Sichtfeld eingeblendet bekommen. Dies ermöglicht eine
schnellere und präzisere Fehlerdiagnose, da relevante Informationen, wie
beispielsweise Schaltpläne, technische Datenblätter oder historische Daten, in
Echtzeit angezeigt werden können. Darüber hinaus können AR-gestützte
Wartungsanleitungen und -simulationen den Technikern helfen, komplexe
Reparaturen oder Wartungsarbeiten durchzuführen, indem sie visuelle
Hilfestellungen und Schritt-für-Schritt-Anleitungen bereitstellen.
\cite{4079262} Dies reduziert das Risiko von Fehlern und verkürzt die
Ausfallzeiten von Maschinen oder Anlagen. Durch die Integration von AR in den
Wartungsprozess können Arbeiter effizienter arbeiten und die cognitive
Arbeitslast reduziert werden \cite{5620905}.

\subsection{Qualitätskontrolle und Inspektion}
Die Anwendung von AR in der Qualitätskontrolle und Inspektion bietet
vielfältige Vorteile, um Prüfprozesse effizienter und präziser zu gestalten. AR
ermöglicht es Inspekteuren, eine optimale Version des fertigen Produktes als
Vergleich einzublenden. Dadurch können Inspekteure Abweichungen oder Mängel
leicht erkennen und bewerten. Dies trägt zur Reduzierung menschlicher Fehler
und zur Verbesserung der Genauigkeit bei der Inspektion bei. Darüber hinaus
kann AR dazu beitragen, den Prozess der Qualitätskontrolle zu beschleunigen.
Hierfür werden potentiell fehlerhafte Bereiche durch Bildverarbeitung oder
Deep-Learning erkannt und dür den Nutzer markiert \cite{9112336}. \\Forschung
auf dem Gebiet der AR in der Qualitätskontrolle und Inspektion hat gezeigt,
dass die Nutzung dieser Technologie zu einer höheren Inspektionsgenauigkeit,
einer schnelleren Fehlererkennung und einer verbesserten Effizienz führen kann.
Die Integration von AR in diesen Bereich bietet somit großes Potenzial, die
Qualitätssicherung in verschiedenen Industriezweigen zu optimieren und die
Inspektionsprozesse zu verbessern.\cite{etonam2019augmented}