\section{Potenzial von AR für neue Geschäftsmodelle}

Neben den bestehenden Anwendungsfällen von AR in der Industrie ergeben sich
durch den stetigen technologischen Fortschritt neue potentielle Anwendungsfälle
und Geschäftsmodelle. In diesem Kapitel werden einige der interessantesten
Potentiale von AR in der Industrie dargestellt.

\subsection{Produktdesign und -entwicklung}
Augmented Reality (AR) bietet ein großes Potenzial für den Anwendungsbereich
des Produktdesigns. Durch die Integration von AR-Technologien in den
Designprozess können Designer und Ingenieure hochwertige Vorschauen und
interaktive Simulationen ihrer Produkte erstellen. Die Nutzung einer
cloud-basierten Anwendung ermöglicht Echtzeit-Kollaboration und Meetings
zwischen Designern weltweit, unabhängig von ihrem Standort. Dies fördert die
Zusammenarbeit, verbessert die Kommunikation und ermöglicht präzisere
Entscheidungen während des Designprozesses. Mit AR können virtuelle Prototypen
in der realen Welt platziert werden, um das Aussehen, die Funktionalität und
die Benutzererfahrung zu bewerten. Dies hilft dabei, Designkonflikte und
-probleme frühzeitig zu erkennen und zu lösen. Darüber hinaus ermöglicht AR
eine immersive Darstellung von Produkten, sodass Kunden und Stakeholder sie vor
dem eigentlichen Bau oder der Produktion in einem realistischen Kontext erleben
können. Dies verbessert nicht nur das Verständnis des Produkts, sondern
ermöglicht auch wertvolles Feedback und iteratives Design. Durch die Nutzung
von AR im Produktdesign können Unternehmen die Effizienz steigern, die
Fehlerquote verringern und letztendlich bessere Produkte entwickeln, die den
Bedürfnissen der Kunden entsprechen.\cite{mourtzis2020augmented}

\subsection{Montage}
Die Nutzung von Augmented Reality (AR) zur Unterstützung von Wartungsarbeiten
an industriellen Geräten hat sich als vielversprechender Anwendungsbereich
erwiesen. Ein weiterer Schritt in dieser Entwicklung wäre der verstärkte
Einsatz von AR in der Produktmontage. Im Vergleich zu gedruckten Handbüchern
bietet AR klare Vorteile, da sie die Aufmerksamkeit auf das zu bearbeitende
Objekt lenkt und Fehler erkennt und reduziert. Frühe Arbeiten auf diesem Gebiet
waren experimentell und konzentrierten sich auf spezifische Objekte. Dabei
wurden häufig Tracking-Techniken wie fiduziale Marker verwendet. Durch
Fortschritte in Bereichen wie Deep Learning kann AR jedoch zunehmend komplexe
Montageprozesse unterstützen. Dennoch ist es teilweise schwierig, die konkreten
Vorteile von AR in diesem Bereich nachzuweisen, und Studien kommen zu
unterschiedlichen Ergebnissen \cite{tang2003comparative}. Dies kann auf
verschiedene Faktoren zurückzuführen sein, wie z.B. die Komplexität der
Montageprozesse, die Qualität der eingesetzten AR-Anwendungen und die
Einarbeitung der Mitarbeiter in die Nutzung der Technologie.
Trotzdem wird die kontinuierliche technologische Weiterentwicklung AR in
Zukunft noch leistungsfähiger machen und den Unternehmen noch mehr
Möglichkeiten bieten, komplexe Montageprozesse zu unterstützen und zu
optimieren. Es ist zu erwarten, dass AR in der Industrie einen immer größeren
Stellenwert einnehmen wird und zu einer effektiven Unterstützung bei der
Montage von Produkten wird.\cite{8951930}

\subsection{AR als Grundlage für neue Produkte und Services}
AR-Systeme können auch als Grundlage für neue Produkte und Services dienen. Ein
Beispiel hierfür ist die Verwendung von AR in der Produktdesign-Phase. Durch
die Nutzung von AR-Systemen können Designer und Ingenieure virtuelle Prototypen
von Produkten erstellen und diese in einer realistischen Umgebung testen.
Dadurch können Fehler vermieden und die Zeit bis zur Markteinführung verkürzt
werden. Ein weiteres Beispiel sind AR-basierte Services. Hierbei können
Unternehmen AR-Brillen oder mobile Geräte an Kunden vermieten, die ihnen dabei
helfen, ihre Produkte zu warten oder zu reparieren. Dieser Service kann für
Kunden besonders attraktiv sein, da er ihnen Zeit und Geld sparen kann.

\subsection{Personalisierte Produktanpassung}
AR kann auch Unternehmen dabei unterstützen, personalisierte Produktanpassungen
anzubieten, was zu neuen Geschäftsmodellen führen kann. Durch die Integration
von AR in den Produktkonfigurationsprozess können Kunden ihre eigenen Produkte
individuell anpassen und in Echtzeit sehen, wie das Endprodukt aussehen wird.
Dies ermöglicht Unternehmen, maßgeschneiderte Produkte anzubieten und
Kundenbedürfnisse besser zu erfüllen.

\subsection{Datenanalyse und Predictive Maintenance}
AR kann Unternehmen auch bei der Datenanalyse und Predictive Maintenance
unterstützen. Durch die Integration von AR in IoT-fähige Geräte und Maschinen
können Unternehmen in Echtzeit Daten von Sensoren erfassen und analysieren, um
frühzeitig Anzeichen von Verschleiß oder Ausfällen zu erkennen. Dies ermöglicht
Unternehmen, Wartungsarbeiten proaktiv zu planen und zu optimieren, um
Ausfallzeiten zu minimieren und die Produktivität zu steigern.

Der Einsatz von Augmented Reality (AR) in der Industrie bietet zahlreiche
Vorteile, aber es gibt auch eine Reihe von Herausforderungen, die bei der
Implementierung und Nutzung von AR-Technologien berücksichtigt werden müssen.
In diesem Kapitel werden einige der wichtigsten Herausforderungen identifiziert
und mögliche Lösungsansätze präsentiert.

Eine der zentralen Herausforderungen besteht in der Integration von AR in
bestehende Arbeitsprozesse und Systeme. Industrielle Umgebungen sind oft
komplex und erfordern eine nahtlose Integration von AR in bestehende Maschinen,
Ausrüstungen und Informationssysteme. Eine Lösung besteht darin,
standardisierte Schnittstellen und Protokolle zu entwickeln, die eine
reibungslose Kommunikation zwischen AR-Systemen und vorhandenen Infrastrukturen
ermöglichen. Die enge Zusammenarbeit zwischen AR-Entwicklern, IT-Spezialisten
und den verschiedenen Fachbereichen in einem Unternehmen ist entscheidend, um
eine erfolgreiche Integration zu gewährleisten.