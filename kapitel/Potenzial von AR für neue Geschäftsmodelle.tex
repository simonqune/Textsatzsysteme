
\section{Potenzial von AR für neue Geschäftsmodelle}
\subsection{Produktvisualisierung und Kundenerlebnis}
Augmented Reality bietet Unternehmen die Möglichkeit, Produkte oder
Dienstleistungen auf innovative Weise zu visualisieren und den Kunden ein
einzigartiges Kundenerlebnis zu bieten. Zum Beispiel können Einzelhändler
AR-Anwendungen nutzen, um virtuelle Anprobe von Kleidung oder virtuelle
Raumgestaltung anzubieten, bei denen Kunden mithilfe von AR-Technologie
realistische 3D-Modelle von Produkten in ihrer eigenen Umgebung sehen können.
Dies ermöglicht den Kunden, Produkte vor dem Kauf zu erleben und trägt dazu
bei, Kaufentscheidungen zu unterstützen und das Kundenerlebnis zu verbessern.

\subsection{Verbesserte Arbeitsprozesse und Effizienz}
AR kann auch dazu beitragen, Arbeitsprozesse und Effizienz in der Industrie zu
verbessern, was zu neuen Geschäftsmodellen führen kann. Durch die Integration
von AR in Arbeitsabläufe können Mitarbeiter visuelle Anweisungen in Echtzeit
erhalten, die die Durchführung von Aufgaben erleichtern und Fehler minimieren
können. Dies kann zu einer Steigerung der Produktivität und einer Verringerung
von Ausschuss oder Ausschusskosten führen. Darüber hinaus können Unternehmen AR
nutzen, um Remote-Schulungen und Schulungsprogramme anzubieten, bei denen
Mitarbeiter in virtuellen Umgebungen geschult werden können, ohne physisch
anwesend zu sein. Dies kann Schulungskosten reduzieren und die
Mitarbeiterentwicklung verbessern.

\subsection{Erweiterte Serviceleistungen und Kundenbindung}
AR ermöglicht es Unternehmen, erweiterte Serviceleistungen anzubieten, die zur
Kundenbindung beitragen können. Zum Beispiel können Unternehmen AR nutzen, um
Echtzeit-Unterstützung und Wartung für ihre Produkte anzubieten. Techniker oder
Kunden können mithilfe von AR-Brillen oder mobilen AR-Geräten auf Anleitungen,
Anweisungen oder Wartungshinweise zugreifen, um Probleme zu beheben oder Fragen
zu klären. Dies kann zu einer verbesserten Kundenzufriedenheit führen und die
Kundenbindung stärken.

\subsection{Personalisierte Produktanpassung}
AR kann auch Unternehmen dabei unterstützen, personalisierte Produktanpassungen
anzubieten, was zu neuen Geschäftsmodellen führen kann. Durch die Integration
von AR in den Produktkonfigurationsprozess können Kunden ihre eigenen Produkte
individuell anpassen und in Echtzeit sehen, wie das Endprodukt aussehen wird.
Dies ermöglicht Unternehmen, maßgeschneiderte Produkte anzubieten und
Kundenbedürfnisse besser zu erfüllen.

\subsection{Datenanalyse und Predictive Maintenance}
AR kann Unternehmen auch bei der Datenanalyse und Predictive Maintenance
unterstützen. Durch die Integration von AR in IoT-fähige Geräte und Maschinen
können Unternehmen in Echtzeit Daten von Sensoren erfassen und analysieren, um
frühzeitig Anzeichen von Verschleiß oder Ausfällen zu erkennen. Dies ermöglicht
Unternehmen, Wartungsarbeiten proaktiv zu planen und zu optimieren, um
Ausfallzeiten zu minimieren und die Produktivität zu steigern.