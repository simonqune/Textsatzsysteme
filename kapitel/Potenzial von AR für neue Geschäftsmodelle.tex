\section{Potenzial von AR für neue Geschäftsmodelle}

Neben den bestehenden Anwendungsfällen von AR in der Industrie ergeben sich
durch den stetigen technologischen Fortschritt neue potentielle Anwendungsfälle
und Geschäftsmodelle. In diesem Kapitel werden einige der interessantesten
Potentiale von AR in der Industrie dargestellt.

\subsection{Produktdesign und -entwicklung}
Augmented Reality (AR) bietet ein großes Potenzial für den Anwendungsbereich
des Produktdesigns. Durch die Integration von AR-Technologien in den
Designprozess können Designer und Ingenieure hochwertige Vorschauen und
interaktive Simulationen ihrer Produkte erstellen. Die Nutzung einer
cloud-basierten Anwendung ermöglicht Echtzeit-Kollaboration und Meetings
zwischen Designern weltweit, unabhängig von ihrem Standort. Dies fördert die
Zusammenarbeit, verbessert die Kommunikation und ermöglicht präzisere
Entscheidungen während des Designprozesses. Mit AR können virtuelle Prototypen
in der realen Welt platziert werden, um das Aussehen, die Funktionalität und
die Benutzererfahrung zu bewerten. Dies hilft dabei, Designkonflikte und
-probleme frühzeitig zu erkennen und zu lösen. Darüber hinaus ermöglicht AR
eine immersive Darstellung von Produkten, sodass Kunden und Stakeholder sie vor
dem eigentlichen Bau oder der Produktion in einem realistischen Kontext erleben
können. Dies verbessert nicht nur das Verständnis des Produkts, sondern
ermöglicht auch wertvolles Feedback und iteratives Design. Durch die Nutzung
von AR im Produktdesign können Unternehmen die Effizienz steigern, die
Fehlerquote verringern und letztendlich bessere Produkte entwickeln, die den
Bedürfnissen der Kunden entsprechen.\cite{mourtzis2020augmented}

\subsection{Wartung und Montage}
Die Nutzung von Augmented Reality (AR) zur Unterstützung von Wartungsarbeiten
an industriellen Geräten hat sich als vielversprechender Anwendungsbereich
erwiesen. Durch die Integration von AR in die Fernwartung könnten sich sogar
noch effektivere Möglichkeiten eröffnen. Bereits erste funktionierende Ansätze
zeigen, dass AR eine bedeutende Rolle bei der Durchführung von
Fernwartungsarbeiten spielen kann. \cite{masoni2017supporting} \\Ein weiterer
Schritt in dieser Entwicklung wäre der verstärkte Einsatz von AR in der
Produktmontage. Im Vergleich zu gedruckten Handbüchern bietet AR klare
Vorteile, da sie die Aufmerksamkeit auf das zu bearbeitende Objekt lenkt und
Fehler erkennt und reduziert. Frühe Arbeiten auf diesem Gebiet waren
experimentell und konzentrierten sich auf spezifische Objekte. Dabei wurden
häufig Tracking-Techniken wie fiduziale Marker verwendet. Durch Fortschritte in
Bereichen wie Deep Learning kann AR jedoch zunehmend komplexe Montageprozesse
unterstützen. Dennoch ist es teilweise schwierig, die konkreten Vorteile von AR
in diesem Bereich nachzuweisen, und Studien kommen zu unterschiedlichen
Ergebnissen \cite{tang2003comparative}. Dies kann auf verschiedene Faktoren
zurückzuführen sein, wie z.B. die Komplexität der Montageprozesse, die Qualität
der eingesetzten AR-Anwendungen und die Einarbeitung der Mitarbeiter in die
Nutzung der Technologie. Trotzdem wird die kontinuierliche technologische
Weiterentwicklung AR in Zukunft noch leistungsfähiger machen und den
Unternehmen noch mehr Möglichkeiten bieten, komplexe Montageprozesse zu
unterstützen und zu optimieren. Es ist zu erwarten, dass AR in der Industrie
einen immer größeren Stellenwert einnehmen wird und zu einer effektiven
Unterstützung bei der Montage von Produkten wird.\cite{8951930}

\subsection{Marketing}
Die Potenziale von Augmented Reality (AR) im Bereich des Marketings sind
vielfältig und vielversprechend. AR ermöglicht es Unternehmen, interaktive und
immersivere Markenerlebnisse für ihre Zielgruppe zu schaffen. Durch die
Einbindung digitaler Inhalte in die reale Welt können Produkte und
Dienstleistungen auf innovative und ansprechende Weise präsentiert werden. AR
bietet die Möglichkeit, Kunden in ihren individuellen Entscheidungskontexten
anzusprechen und personalisierte, relevante Informationen bereitzustellen.
Darüber hinaus eröffnet AR neue Wege der Kundeninteraktion, indem es
spielerische Elemente integriert. Durch die Schaffung von emotionalen Bindungen
und positiven Markenerlebnissen kann AR das Kundenengagement und die
Markenloyalität erhöhen. Unternehmen haben auch die Möglichkeit, AR als
Instrument für datengesteuertes Marketing einzusetzen, indem sie Nutzungsdaten
und Interaktionsmuster erfassen und analysieren, um gezielte Marketingmaßnahmen
abzuleiten. Die kontinuierlichen Fortschritte in der AR-Technologie und die
steigende Verbreitung von AR-fähigen Endgeräten eröffnen spannende Perspektiven
für innovative Marketingstrategien und -kampagnen, die das Kundenerlebnis
revolutionieren können.\cite{chylinski2020augmented,rauschnabel2019augmented}

\subsection{Schulung und Training}
Augmented Reality (AR) wird immer wieder im Zusammenhang mit Schulungsprozessen
in der Industrie erwähnt. Besonders interessant ist dabei der Einsatz von
AR-basierten Assistenzsystemen während des Schulungsprozesses. Theoretisch
könnten dadurch der Umgang mit potenziell gefährlichen Situationen in einer
sicheren Umgebung trainiert werden. Auch das Erlernen von Montageaufgaben ist
mit AR-Unterstützung möglich. Dabei können Anleitungen in Papierform durch AR
ersetzt und Fehler beim Aufbau durch den Monteur erkannt und mitgeteilt werden.
Allerdings gibt es in diesem Bereich keine klaren Studien, die den besseren
Lernerfolg durch AR belegen. Am besten schneidet weiterhin die persönliche
Schulung ab. Das AR-Assistenzsystem verhindert effektiv ein fehlerhaftes
Erlernen von Inhalten, erreicht jedoch nicht die Geschwindigkeit der
persönlichen Schulung. Im Vergleich zu einer individuellen Schulung mit einem
gedruckten Handbuch konnte keine höhere Effizienz der AR-Schulung nachgewiesen
werden. Bezüglich der langfristigen Wissensnachhaltigkeit zeigt sich, dass es
keine Unterschiede im Erinnerungsvermögen der Teilnehmer an den Montageprozess
gab, unabhängig von der Schulungsmethode. Die Ergebnisse legen nahe, dass
AR-Assistenzsysteme als hilfreiche Werkzeuge für die Schulung von Arbeitern in
Montageaufgaben eingesetzt werden können, insbesondere um fehlerhaftes Lernen
zu vermeiden. Es ist jedoch zu beachten, dass weitere Vorteile wie eine höhere
Schulungseffizienz in dieser Studie nicht belegt werden konnten. Dennoch ist zu
erwarten, dass die Effizienz der Schulungen durch den technischen Fortschritt
im Bereich AR weiter zunehmen wird und AR-gestützte Schulungen Unternehmen
einen großen Vorteil bieten können.