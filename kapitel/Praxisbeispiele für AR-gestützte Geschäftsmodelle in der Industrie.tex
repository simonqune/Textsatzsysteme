\section{Praxisbeispiele für AR-gestützte Geschäftsmodelle in der Industrie
 }

AR-Technologie kann auch in der Qualitätskontrolle und Inspektion eingesetzt
werden, um Fehler zu minimieren und die Produktivität zu erhöhen. Durch die
Verwendung von AR können Mitarbeiter beispielsweise Mustererkennungssysteme
verwenden, um Abweichungen von den vorgegebenen Standards zu erkennen.
AR-Brillen können auch mit Kameras ausgestattet werden, um Inspektionsprozesse
zu automatisieren und zu beschleunigen.

\subsection{Wartung und Instandhaltung}
Die Nutzung von AR in der Wartung und Instandhaltung hat sich als äußerst
effektiv erwiesen. Durch die Verwendung von AR-Brillen können Techniker
beispielsweise Informationen zu den Geräten direkt auf dem Bildschirm
einblenden lassen, was die Reparaturzeit verkürzt und die Effizienz erhöht.
Einige Unternehmen setzen bereits AR-Technologie zur Unterstützung der Wartung
und Instandhaltung von Flugzeugen, Schiffen und anderen komplexen Systemen ein.

\subsection{Schulung und Training}
AR-Technologie bietet auch enorme Vorteile im Bereich Schulung und Training.
Durch die Verwendung von AR können Mitarbeiter beispielsweise in einer
virtuellen Umgebung trainiert werden, was zu einer höheren Effizienz und
Genauigkeit führt. AR-Brillen können auch verwendet werden, um komplexe
Prozesse und Arbeitsabläufe zu simulieren und den Mitarbeitern praktische
Erfahrungen zu vermitteln, bevor sie in der realen Welt eingesetzt werden.

\subsection{Qualitätskontrolle und Inspektion}
AR-Technologie kann auch in der Qualitätskontrolle und Inspektion eingesetzt
werden, um Fehler zu minimieren und die Produktivität zu erhöhen. Durch die
Verwendung von AR können Mitarbeiter beispielsweise Mustererkennungssysteme
verwenden, um Abweichungen von den vorgegebenen Standards zu erkennen.
AR-Brillen können auch mit Kameras ausgestattet werden, um