\section{Fazit}

In dieser Arbeit wurde untersucht, wie die Anwendung von Augmented Reality (AR)
in der Industrie zur Entwicklung neuer Geschäftsmodelle beitragen kann. Es
werden verschiedene Anwendungsfälle von AR in der Industrie diskutiert, wie
beispielsweise Wartung und Instandhaltung, Qualitätskontrolle und Inspektion,
Produktdesign und -entwicklung, Montage und Marketing.

In Bezug auf die Wartung und Instandhaltung kann AR die Effizienz und
Genauigkeit dieser Prozesse signifikant verbessern. Techniker können visuelle
Informationen und Anweisungen direkt in ihr Sichtfeld eingeblendet bekommen,
was eine schnellere und präzisere Fehlerdiagnose ermöglicht.

AR bietet auch ein großes Potenzial für den Anwendungsbereich des
Produktdesigns. Durch die Integration von AR-Technologien in den Designprozess
können Designer und Ingenieure hochwertige Vorschauen und interaktive
Simulationen ihrer Produkte erstellen.

Die Nutzung von AR zur Unterstützung von Wartungsarbeiten an industriellen
Geräten hat sich als vielversprechender Anwendungsbereich erwiesen. Durch die
Integration von AR in die Fernwartung könnten sich sogar noch effektivere
Möglichkeiten eröffnen.

Im Bereich des Marketings sind die Potenziale von AR vielfältig und
vielversprechend. AR ermöglicht es Unternehmen, interaktive und immersivere
Markenerlebnisse für ihre Zielgruppe zu schaffen.

Trotz der enormen Fortschritte im Bereich der Augmented Reality in den letzten
Jahren, bestehen weiterhin einige Herausforderungen, die es zu bewältigen gilt.
Dazu gehören präzise und zuverlässige Tracking-Methoden, Herausforderungen im
Bereich der AR-Darstellung und Datenschutzbedenken.

Insgesamt zeigt das Paper, dass AR eine vielversprechende Technologie ist, die
das Potenzial hat, die Industrie zu revolutionieren und neue Geschäftsmodelle
zu ermöglichen. Es wird erwartet, dass AR in der Industrie einen immer größeren
Stellenwert einnehmen wird und zu einer effektiven Unterstützung bei der
Montage von Produkten wird.