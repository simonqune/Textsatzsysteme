\section{Einführung}

In den letzten Jahren hat Augmented Reality (AR) sowohl allgemein, als auch in
der Industrie zunehmend an Bedeutung gewonnen. AR ermöglicht es, digitale
Inhalte in die physische Welt zu projizieren und somit die Realität mit
virtuellen Informationen und Objekten zu erweitern. Diese Technologie eröffnet
weitreichende Möglichkeiten zur Optimierung von Prozessen, Steigerung der
Produktivität und Entwicklung innovativer Geschäftsmodelle in der Industrie.

Der Einsatz von AR in der Industrie ist bereits weit verbreitet und vielfältig.
Er findet Anwendung beim Training und der Schulung von Mitarbeitern, bei der
Wartung und Instandhaltung von Maschinen und Anlagen sowie bei der
Produktentwicklung und Optimierung von Produktionsprozessen. Dabei kommen
verschiedene AR-Technologien und Plattformen zum Einsatz, wie beispielsweise
Head-mounted Displays (HMDs), Smart Glasses oder markerbasierte AR-Systeme.