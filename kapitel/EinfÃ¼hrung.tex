\section{Einführung}

Dank des technologischen Fortschritts im Bereich der Augmented Reality wurden
enorme Fortschritte erzielt. AR ermöglicht es, digitale Inhalte in die
physische Welt zu projizieren und somit die Realität mit virtuellen
Informationen und Objekten zu erweitern. Diese Technologie eröffnet
weitreichende Möglichkeiten zur Optimierung von Prozessen, Steigerung der
Produktivität und Entwicklung innovativer Geschäftsmodelle in der Industrie.

Der Einsatz von AR in der Industrie ist bereits weit verbreitet und vielfältig.
Er findet Anwendung bei der Wartung und Instandhaltung von Maschinen und
Anlagen sowie bei der Qualitätskontrole. Dabei kommen verschiedene
AR-Technologien und Plattformen zum Einsatz, wie beispielsweise Head-mounted
Displays (HMDs), Smart Glasses oder markerbasierte AR-Systeme. Diese
Technologien bieten den Benutzern ein immersives AR-Erlebnis und ermöglichen
eine direkte Interaktion mit den virtuellen Inhalten in der realen Umgebung.\\